\documentclass{EPL-master-thesis-covers-EN}

% --------------------------------------------------
%                 PACKAGES
% --------------------------------------------------
\usepackage{csquotes}              % Recommended with biblatex
\usepackage{graphicx}              % Include graphics
\usepackage{subcaption}            % Subfigures
\usepackage{float}                 % Improved float control
\usepackage{booktabs}              % Better tables
\usepackage{array}                 % More table options
\usepackage{amsmath,amssymb,amsfonts} % Math
\usepackage{siunitx}               % Units
\usepackage[hidelinks]{hyperref}   % Clickable links (no colors)
\usepackage{url}                   % Better URL handling
\usepackage{enumitem}              % Control over lists
\usepackage{setspace}              % Line spacing
\usepackage{caption}               % Caption control
\usepackage{listings}              % Code listings
\usepackage{xcolor}                % Colors for code
\usepackage{svg}
% --------------------------------------------------
%      TABLE OF CONTENTS DEPTH CONFIGURATION
% --------------------------------------------------
\setcounter{tocdepth}{3}      % Show up to subsubsections in ToC
\setcounter{secnumdepth}{3}   % Number up to subsubsections

% --------------------------------------------------
%            BIBLIOGRAPHY (BIBLATEX)
% --------------------------------------------------
\usepackage[
    backend=biber,
    style=numeric,
    sorting=none,
    maxbibnames=99
]{biblatex}

\addbibresource{refs.bib}

% --------------------------------------------------
%            GRAPHICS PATH
% --------------------------------------------------
\graphicspath{{figures/}}

% --------------------------------------------------
%            LISTINGS CONFIG (CODE)
% --------------------------------------------------
\lstset{
    basicstyle=\ttfamily\small,
    numbers=left,
    numberstyle=\tiny,
    stepnumber=1,
    numbersep=5pt,
    showstringspaces=false,
    frame=single,
    breaklines=true,
    tabsize=2,
    captionpos=b,
    commentstyle=\color{gray},
    keywordstyle=\color{blue},
    stringstyle=\color{red}
}

% --------------------------------------------------
%            DOCUMENT METADATA (EPL FORMAT)
% --------------------------------------------------
% Title and subtitle
\title{Smartphone positioning accuracy in forensic contexts}
\subtitle{Satellite and cellular methods across Belgian environments}

% Author (use \textsc{} for last name)
\author{Romain \textsc{Mottet}}
% \secondauthor{Second Author \textsc{Name}} % Uncomment if needed
% \thirdauthor{Third Author \textsc{Name}}   % Uncomment if needed

% Degree title (copy exactly from EPL list)
\degreetitle{Master [120] in Computer Science and Engineering}

% Supervisor(s)
\supervisor{Ramin \textsc{Sadre}}
% \secondsupervisor{Second Supervisor \textsc{Name}} % Uncomment if needed
% \thirdsupervisor{Third Supervisor \textsc{Name}}   % Uncomment if needed

% Readers
\readerone{\textsc{Corentin Libert}}
\readertwo{\textsc{Yinan Cao}}     % Uncomment if needed
% \readerthree{Reader Three \textsc{Name}} % Uncomment if needed
% \readerfour{Reader Four \textsc{Name}}   % Uncomment if needed
% \readerfive{Reader Five \textsc{Name}}   % Uncomment if needed

% Academic year
\years{2025--2026}

% --------------------------------------------------
%            BEGIN DOCUMENT
% --------------------------------------------------
\begin{document}

% Front cover page (uses \maketitle for EPL class)
\maketitle

% Roman numbering for front matter
\pagenumbering{roman}

% --------------------------------------------------
%                ABSTRACT
% --------------------------------------------------
\chapter*{Abstract}
\addcontentsline{toc}{chapter}{Abstract}

This Master’s thesis validates smartphone geolocation reliability across forensic scenarios 
in urban (Ixelles), suburban (Louvain-la-Neuve), and rural (Waha) Belgium. It 
compares smartphone GNSS coordinates against two independent ground truth 
references: a smartwatch (Garmin Venu Sq 2) and physical waypoints marked by 
timestamped photographs. The thesis addresses two research questions: (RQ1) How 
accurate is smartphone GNSS positioning relative to ground truth? and (RQ2) Can 
cellular LTE signal strength reconstruct device locations without GPS? The 
methodology independently validates satellite-based positioning (RQ1) through 
comparative accuracy analysis, and develops network-based positioning methods 
(RQ2) via empirical distance models and trilateration. This two-pronged approach 
reflects forensic practice: GPS logs provide evidence of outdoor presence, while 
LTE-based reconstruction serves scenarios where satellite signals are unavailable.


\vspace{1em}
\noindent\textbf{Open science resources}\\
All research materials are available under the MIT license on GitHub. The repository includes analytical code, the complete experimental dataset, and the \LaTeX{} source of this document. Smartphone image files are not hosted due to their 4\,GB size each, but anyone needing them can contact me directly.

\begin{center}
  \url{https://github.com/romain-mottet/forensic-gnss-lte-analysis}
\end{center}

Researchers are invited to verify these results or extend the codebase for future forensic work.
\clearpage

% --------------------------------------------------
%                ACKNOWLEDGEMENTS
% --------------------------------------------------
\chapter*{Acknowledgements}
\addcontentsline{toc}{chapter}{Acknowledgements}

I thank Professor Ramin Sadre for his guidance, expertise, and consistent support during this research. His feedback and critiques shaped the method, interpretation, and final quality of this thesis.

I thank Jean-Yves Lepage for facilitating contact with the Police Judiciaire Fédérale de Charleroi and for sharing essential technical insights that supported this work.

I also thank my family and friends for their constant encouragement and patience throughout the project. Their support kept me focused and motivated through demanding stages of the research.

This thesis acknowledges the use of Generative AI (GAI) tools for language refinement and grammar correction, in line with Université Catholique de Louvain (UCLouvain) and École Polytechnique de Louvain (EPL) guidelines. The tools served only to improve readability. All ideas, analyses, and conclusions are entirely my own.

\clearpage

% --------------------------------------------------
%                TABLE OF CONTENTS
% --------------------------------------------------
\tableofcontents
\clearpage

% Switch to Arabic numbering for main matter
\pagenumbering{arabic}

% --------------------------------------------------
%                CHAPTERS
% --------------------------------------------------
\onehalfspacing % 1.5 line spacing for readability

\chapter{Introduction}
\label{ch:introduction}

Smartphones now serve as the main geolocation tools worldwide, replacing dedicated GNSS receivers in daily use and forensic investigations. This thesis evaluates the reliability of smartphone positioning in police investigations, focusing on accuracy, precision, and consistency across different Belgian environments.

The research combines controlled field measurements and signal analysis. It measures GNSS performance under varying environmental conditions and complements satellite data with LTE signal strength models to reconstruct locations without GPS. These methods reflect real forensic conditions where incomplete or degraded positioning data often occur.

Geolocation quality depends on GNSS geometry, signal obstruction, multipath effects, sensor integration, and operating system algorithms such as those on Android devices. This introduction outlines the motivation, forensic relevance, research objectives, and structure of the thesis.

\section{Motivation}
\label{sec:intro_motivation}

Navigation, fitness tracking, and location-based applications rely on smartphone positioning. In digital forensics, police increasingly use smartphone location data to confirm presence at crime scenes, assess alibis, and analyze movement patterns. Yet, clarity about actual accuracy across different conditions remains limited. Users, engineers, and legal professionals often misjudge what current devices can reliably deliver. Quantifying these limits is essential for designing dependable systems, setting informed user expectations, and defining objective standards for admissible geolocation evidence in Belgian courts.

\subsection{Global context: Smartphone GNSS ubiquity}
\label{subsec:intro_global_context}

Over half the global population owns a smartphone as of 2023~\cite{GSMA2023SOMIC,ExplodingTopics2025}. Nearly 4 billion people access mobile internet via smartphones, accounting for 49\% of the world's population~\cite{GSMA2023SOMIC}. Global smartphone shipments have exceeded 7 billion units, with over 90\% of Android phones supporting raw GNSS measurements since Android 7 (2016)~\cite{GoogleAndroid2025}.

The GNSS chip market reached USD 7.47 billion in 2023 and is projected to reach USD 11.53 billion by 2030 (CAGR 6.6\%)~\cite{PSMarketResearch2024}. Mobile technology contributed USD 6.5 trillion to global GDP in 2024~\cite{GSMA2025MobileEconomy}. Location-based services, entirely dependent on smartphone GNSS, reached USD 46.7 billion in 2024 and are projected to expand at a CAGR of 15.84\%, reaching USD 187 billion by 2033~\cite{IMARCGroup2024LBS}.

Smartphones are systematically replacing dedicated GNSS receivers, with the handheld GPS receiver market projecting much slower growth: from USD 1.2 billion in 2024 to USD 2.4 billion by 2033~\cite{Marketintelo2025}. The European Commission mandates GNSS positioning for new smartphones, with over 95\% of chipsets supporting Galileo~\cite{MordorIntelligence2024}.

Performance gaps between smartphones and professional surveying equipment remain substantial. Smartphone GNSS in forests yields RMSE of 4.5-11.4~m, compared to 1.2-1.4~m for survey-grade receivers~\cite{Tomastik2017}. Open-sky conditions improve smartphone error to 2.1~m, while survey receivers achieve 0.06~m~\cite{Tomastik2017}. Dual-frequency smartphones in kinematic mode report RMS errors of 3.2-4.9~m~\cite{Robustelli2021}. High-end models achieve approximately 4~m error, outperforming low-end models at 6~m~\cite{Petrella2025}.

\subsection{Applications of smartphone geolocation}
\label{subsec:intro_applications}

Modern smartphones form the core of many location-based systems:
\begin{itemize}[noitemsep]
    \item Navigation and mapping tools
    \item Fitness and health tracking
    \item Ride-hailing and delivery platforms
    \item Augmented reality applications
    \item Social and contextual networking services
\end{itemize}

Each field demands a different level of accuracy. Commercial navigation functions effectively with errors of 3--5~m. Forensic work operates under stricter constraints. A 5~m deviation can decide whether location data confirm or contradict an alibi. Placing a device 8~m from a crime scene differs markedly from locating it 12~m away when assessing evidentiary consistency. This precision requirement defines the forensic threshold and drives the motivation for this research.

\subsection{Challenges in real-world scenarios}
\label{subsec:intro_challenges}

Smartphone geolocation faces several practical challenges that degrade quality in operational environments.

\subsubsection{Signal obstruction and multipath}

Multipath occurs when satellite signals reach a receiver via direct and reflected paths simultaneously. Reflected signals bouncing off buildings, ground surfaces, and metallic structures interfere with direct signals, causing inaccurate position calculations. Code measurements are most vulnerable: intersecting direct and reflected signals cause 2--3~m errors, while only-reflected signals can cause errors of hundreds of meters~\cite{Peretic2025,Weng2023}. Multipath severity increases with urban canyon depth and symmetry. In urban canyons with tall buildings on both sides, multipath degradation is way worse than in asymmetric canyons~\cite{Peretic2025}.

Signal obstruction is environmental and predictable. Open-sky environments exhibit minimal multipath. Suburban areas experience mild multipath. Urban environments (``urban canyons'') with tall buildings cause significant multipath leading to the largest errors~\cite{Peretic2025}.

\subsubsection{Antenna design and device form factor}

Smartphones are constrained by weight, cost, and aesthetic considerations. They typically incorporate low-cost GNSS receivers and antennas, increasing positioning errors compared to professional surveying equipment, which employs precisely engineered antenna phase centers and optimized ground planes. Optimal antenna shielding through ground plane design minimizes multipath: the optimal ground plane is approximately 1.5 times the signal wavelength. Smartphone design rarely accommodates such optimization due to form-factor trade-offs~\cite{Weng2023}.

The human body itself becomes an obstruction. The user holding a smartphone blocks a portion of incoming GNSS signal. Smartphone antenna phase center location relative to device structure affects position accuracy, with largest variations aligned with major components (housing, active electronics)~\cite{Weng2023}.

\subsubsection{Environmental variability and temporal factors}

Sky visibility (the proportion of unobstructed sky) is a primary predictor of GNSS accuracy. Building heights, street widths, tree canopy coverage, time-of-day (affecting satellite geometry), and weather conditions (cloud cover, humidity) introduce temporal and environmental variability. Network-assisted positioning (A-GPS, SUPL) improves accuracy by 1--2~m in urban areas by downloading satellite orbital information from network servers~\cite{Zangenehnejad2021}. Custom ROM configurations (e.g., LineageOS) allow toggling network assistance, restricting GNSS constellations (GPS-only versus multi-constellation), and adjusting update rates, all of which affect reported coordinates.

\subsection{Belgian forensic context: Legal and technical gap}
\label{subsec:intro_belgian_context}

Belgian judges and prosecutors increasingly utilize smartphone geolocation evidence in criminal investigations, yet rigorous standardized studies demonstrating its reliability and accuracy under defined conditions remain lacking~\cite{CrucibleLaw2024,ProvenData2024}. International research provides isolated measurements but no systematic comparison across Belgian urbanization levels. Most studies measure accuracy at single time points; few quantify precision across repeated days (essential for forensic credibility)~\cite{ProvenData2024,CriminalLegalNews2024}.
The Belgian legal context compounds this technical gap. Belgium's data retention legislation has faced three constitutional challenges. The European Court of Justice ruled in 2021 that Belgium's blanket retention scheme violated European law~\cite{EDRI2024}. The Constitutional Court approved a revised 2022 version allowing selective retention in high-crime zones, but the judicial system lacks technical standards for validating accuracy and reliability of retained geolocation data~\cite{EDRI2024}. Legal authorization to collect geolocation data does not address the technical question: is position data accurate enough for forensic evidence?

The 2023 Sky~ECC case demonstrates current practice. Belgian courts convicted suspects partly on location data derived from smartphone positioning without full technical vetting of uncertainty margins, environmental influences, or device reliability~\cite{Oerlemans2023}. The Belgian Data Protection Authority acknowledged that ``accuracy of geolocation data varies substantially depending on environmental factors'' but provided no quantitative guidance for expectations across environments~\cite{Stibbe2024a}. Under emerging European evidence standards (E-evidence Regulation entering force August 2026), cross-border digital evidence must meet rigorous reliability standards~\cite{Stibbe2024b}. Without Belgian-specific GNSS accuracy studies, Belgian courts risk convictions being challenged as insufficiently supported by validated technical evidence.
\section{Research Questions}
\label{sec:intro_research_questions}
This thesis addresses two core research questions:
\begin{itemize}[noitemsep]
    \item \textbf{RQ1: How does smartphone GNSS accuracy compare to smartwatch and physical waypoint references?} This question evaluates the deviation between devices in urban (Ixelles), suburban (Louvain-la-Neuve), and rural (Waha) locations. It identifies which environment produces the most consistent coordinates for forensic reconstruction.
    \item \textbf{RQ2: How effectively can LTE signal strength models estimate location without GPS?} This research develops empirical distance models using network signals to determine a device’s position independently. The analysis measures if these network-based methods provide sufficient precision when satellite signals are obstructed or unavailable.
\end{itemize}
These questions provide the quantitative data police need to verify suspect locations. Investigators must know if a specific coordinate reliably places a person at a crime scene or merely within a general area.
\subsection{Scope and limitations}
\label{subsec:intro_scope}

This study focuses on consumer-grade Android smartphones (OnePlus 6T running LineageOS 22.2 with Android 15) using standard GNSS positioning APIs and GNSS capabilities. It does not cover specialized surveying-grade equipment, differential correction techniques beyond standard GNSS, or proprietary correction services.

The experimental scope encompasses:
\begin{itemize}[noitemsep]
    \item three predefined $\approx$ 800 m walking routes in distinct Belgian urbanization environments
    \item Five repetitions per location
    \item Five predefined measurement stops per route
    \item Horizontal accuracy and precision (repeatability) as primary metrics
    \item Environmental and configuration factors as explanatory variables
\end{itemize}

Reference ground truth is established using a Garmin Venu Sq 2 consumer-grade watch (acknowledged as a study limitation; consumer-grade equipment has 5-10~m accuracy itself, not professional surveying-grade precision).

\section{Thesis structure}
\label{sec:intro_structure}

The remainder of the thesis is organized as follows.

\textbf{Chapter~\ref{ch:state_of_the_art}} reviews the state of the art on smartphone geolocation, GNSS signal processing, multipath effects, antenna design, and forensic evidence admissibility standards. It synthesizes international literature while emphasizing gaps specific to Belgian forensic contexts.

\textbf{Chapter~\ref{ch:experimental_framework}} documents the journey from commercial forensic tools to open-source alternatives, detailing ethical constraints, device selection criteria, and the final hardware configuration. It provides replication guidance for future investigators navigating encryption barriers, firmware availability, and bootloader restrictions across manufacturer platforms.

\textbf{Chapter~\ref{chap:methodology}} describes the three-stage experimental pipeline: GPS comparison against Garmin watch and EXIF waypoints, LTE distance regression through Leave-One-File-Out cross-validation, and network-based trilateration across four environmental contexts. Field protocols cover three Belgian environments (urban Ixelles, suburban Louvain-la-Neuve, rural Waha) with five repetitions per location and automated quality assurance mechanisms.

\textbf{Chapter~\ref{chap:results}} presents GNSS accuracy metrics (CEP, R95, RMSE) stratified by environment and atmospheric conditions, validates LTE distance estimation via signal-to-tower regression, and quantifies trilateration positioning error relative to ground truth. Key findings establish baseline accuracy expectations and confidence intervals for forensic use.

\textbf{Chapter~\ref{ch:conclusion}} synthesizes findings, establishes contributions to Belgian forensic standards, acknowledges experimental limitations, and proposes future research directions for improved positioning reliability and broader device validation.

\newpage
\chapter{State of the Art}
\label{ch:state_of_the_art}

\section{Forensics and digital investigation}
\label{chstateoftheart}


\subsection{Difference between mistake and error}
\label{secsotamistakeerror}

Forensic analysis separates operational failures into two distinct categories: mistakes and errors. A mistake is an unintentional incorrect result generated despite the presence of adequate safeguards \cite{Ryser2024}. The practitioner possesses the necessary knowledge but fails to apply the procedure correctly.

Errors encompass any deviation from the true value. These discrepancies occur regardless of the investigator's skill level or procedural awareness \cite{Casey2020}. Digital forensics demands a rigorous focus on errors, as these deviations dictate the reliability of tools and the ultimate integrity of the evidence.

Quality assurance strategies depend on this classification.
\begin{itemize}
    \item Mistakes can be resolved through training and refined workflows.
    \item Errors represent measurement failures that require systematic investigation and may be irreducible.
\end{itemize}

Recognizing this distinction allows for targeted improvements and ensures accurate communication of forensic findings.



\subsection{Three types of errors}
\label{secsotaerrortypes}

Forensic investigation and measurement include three error categories: systematic, random, and gross errors \cite{Brainkart2018}.

\textbf{Systematic errors:} These errors create consistent inaccuracies in the same direction because of flaws in instruments or procedures \cite{Scribbr2023}. They move measurements away from the true value. In digital forensics, these errors often come from software bugs or tool limitations. Averaging multiple measurements does not fix this problem \cite{Brainkart2018}. Undetected systematic errors reduce accuracy and cause incorrect conclusions.

\textbf{Random errors:} Unpredictable fluctuations in measurement cause random errors \cite{Brainkart2018}. They make data scatter around a mean value and affect precision.
\begin{itemize}
    \item Measurements vary in different directions.
    \item Averaging results reduces the impact of these errors because opposite values cancel each other.
    \item Using large sample sizes helps to minimize these effects \cite{Scribbr2023}.
\end{itemize}

\textbf{Gross errors:} These errors come from human mistakes, such as carelessness or lack of experience \cite{Brainkart2018}. They are usually large and easy to find with quality control. Examples include reading an instrument incorrectly or misinterpreting a result.

In digital forensics, systematic errors happen more often than random errors due to tool design and standard procedures \cite{Nelson2015}. Investigators must find and reduce these errors by using validation tests and quality assurance rules.

\subsection{What is a digital trace?}
\label{secsotadigitaltrace}

A digital trace is any information stored or transmitted in digital form that relates to events under investigation and can be secured, fixed, and decoded using forensic methods \cite{FIRST2019}. Digital traces encompass any changes in digitized environments that reflect activities relevant to investigation, including data objects and physical evidence recovered from digital devices.

Locard's Exchange Principle applies to digital forensics: every contact leaves a trace, whether presence or absence of information \cite{FIRST2019}. Digital traces differ from physical traces in their volatility and complexity. They may be easily modified, deleted, or degraded without detection. Multiple independent formats may contain the same information, providing verification opportunities.

The concept broadens forensic investigation beyond traditional physical evidence. A digital trace can originate from operating system logs, application data, metadata embedded in files, network communications, or hardware-level artifacts. Device interactions systematically generate traces. Understanding trace generation mechanisms, storage locations, and potential degradation pathways proves essential for proper evidence handling.

\subsection{Definition of location trace}
\label{secsotalocationtrace}

A location trace is a digital artifact that records or implies where a device or user was at specific points in time. It can come from systems that explicitly capture position data or from data produced indirectly during normal device use. Forensic analysis relies on such traces to reconstruct movements, build timelines, and check proximity between a person and a place.

Each location trace links spatial coordinates with a timestamp, forming a spatiotemporal record of device positions. Its reliability depends on the accuracy limits of the underlying positioning technology and on any degradation that occurs between creation and forensic extraction. Different positioning methods therefore generate traces with distinct accuracy and reliability profiles \cite{Magnetforensics2025}.

In criminal investigations, location traces can support or challenge alibi statements, associate suspects with crime scenes, and reveal recurring behavioural patterns. Their value as evidence fully depends on how well investigators understand the technical and procedural mechanisms that produced them.

\subsection{Three phases in location trace analysis}
\label{secsotatracephases}

Digital trace analysis follows three stages: production, persistence, and investigation \cite{Berger2024}.

\subsubsection{Trace production}
\label{subsecsotatraceproduction}

Trace production includes the acquisition, processing, and storage of location data by a device. These mechanisms define the accuracy and completeness of the information. Devices often lose context during this stage by failing to record the specific positioning method or its estimated error \cite{Spichiger2023}. Risks include the storage of inaccurate coordinates and time synchronization errors without indicating these failures. Data may also become fragmented or corrupted within the storage structure before it can be retrieved.

\subsubsection{Trace persistence}
\label{subsecsotatracepersistence}

Persistence refers to the recoverability of a trace after its creation. Digital traces are volatile and frequently disappear over time \cite{Berger2024}. This decay can create a false impression of completeness if some traces remain while others are lost.
\begin{itemize}
    \item Missing data results from failures in generation, preservation, or forensic recovery \cite{Berger2024}.
    \item Storage limits and overwriting mechanisms reduce the lifespan of records.
    \item Cloud synchronization and application pruning actively remove traces from local storage.
\end{itemize}
The absence of a location trace differs from proof of absence; it signifies that no data was successfully preserved \cite{Berger2024}.

\subsubsection{Trace investigation}
\label{subsecsotatraceinvestigation}

The investigation phase focuses on extracting and parsing raw data into structured models for analysis \cite{Berger2024}. Success requires an investigator to understand device memory copies and the specific data models used by forensic tools. Because manufacturers frequently change database formats, tools must be regularly validated to ensure they do not ignore or misinterpret traces on newer devices \cite{Berger2024}.

Reconstruction explains the existence of traces through hypothesis testing. Analysts must incorporate position accuracy into their evaluations to avoid reaching erroneous conclusions. Communicating uncertainty clearly at every stage remains essential for a reliable forensic interpretation \cite{Berger2024}.

\subsection{Traces not designed for geolocation}
\label{secsotanonpositioningtraces}

Location information exists within various digital traces that were not primarily created for positioning.

\textbf{EXIF metadata:} Smartphones and digital cameras automatically record EXIF metadata during image capture. These files include technical camera specifications, timestamps, exposure parameters, and geographical coordinates from GNSS \cite{Oikonomidis2019}. Geographic data typically uses a degrees, minutes, and seconds format. Conversion to decimal degrees follows this formula:
\[
\text{decimal degrees} = \text{degrees} + \frac{\text{minutes}}{60} + \frac{\text{seconds}}{3600}
\]
\cite{Oikonomidis2019}. Specialized tools such as ExifTool GUI and Metadata++ extract these geotags. Python libraries like Pillow enable the forensic extraction of location data and technical camera configurations directly from image files \cite{Oikonomidis2019}.

\textbf{Metadata in digital artifacts:} Digital metadata includes descriptive, structural, administrative, and statistical information \cite{Oikonomidis2019}. 
\begin{itemize}
    \item Application data and documents embed timestamps that mark specific events.
    \item These timestamps allow for cross-verification against independent timelines.
    \item Document properties record creation locations or modification times that reveal user patterns.
\end{itemize}

\textbf{Messages and communication records:} Text messages, social media posts, and emails frequently contain location references within the text. Users may explicitly mention addresses or directions in their conversations. Navigation applications also record when a user shares a route or a destination. Social media platforms further encode positioning through check-ins and location tags \cite{Spichiger2023}.

\textbf{Browser history and navigation applications:} Web browsers log visited URLs alongside precise timestamps. Navigation applications maintain extensive records of searched locations, planned routes, and actual paths traveled. Saved favorite locations and specific map views indicate direct user interaction with geographic data. Search queries for directions between two points confirm that a user considered those locations relevant to their activity.


\subsection{Software analysis for mobile device forensics}
\label{secsotasoftwareanalysis}

\subsubsection{Encryption and security models}
\label{subsecsotaencryptionoverview}

Modern smartphones use strong encryption to protect data at rest. Android and iOS implement different security architectures to manage this protection.

\textbf{Android encryption:} Android devices utilize either Full-Disk Encryption (FDE) or File-Based Encryption (FBE). FBE is the required standard for all devices launched with Android 10 or higher \cite{Samsung2024}. While FDE secures the entire user data partition with a single primary key, FBE encrypts individual files using unique keys derived from a primary key \cite{Samsung2024, Android2025}.
\begin{itemize}
\item FBE supports granular security through specific storage locations.
\item Credential-encrypted storage remains locked until the user provides authentication.
\item Device-encrypted storage is available immediately during Direct Boot mode \cite{Samsung2024, Android2025}.
\item The Trusted Execution Environment (TEE) protects cryptographic keys within hardware-backed storage \cite{Technical2025}.
\end{itemize}

\textbf{iOS encryption:} iOS uses filesystem-level encryption that merges a unique hardware key (UID) with the user passcode \cite{Technical2025}. Files are assigned to different Data Protection classes, each providing a specific level of security. A dedicated hardware security processor called the Secure Enclave manages these cryptographic keys and enables biometric authentication \cite{Technical2025}. This processor keeps keys isolated from the main system to prevent software-based attacks. These protections make accessing encrypted data on modern iOS devices difficult for forensic investigators \cite{Technical2025}.

\subsubsection{Commercial vs Open-Source forensic tools}
\label{subsec:tool-comparison}

Forensic teams in Charleroi balance extraction power against strict budget limits. Commercial tools offer speed. They handle encryption without requiring root access. Budget constraints restrict these licenses to a few shared units. Open-source alternatives cost nothing. They require root access. This risks evidence integrity. Charleroi police avoid them completely.

\begin{table}[h]
\centering
\small
\begin{tabular}{|p{2.5cm}|p{4cm}|p{4cm}|p{4cm}|}
\hline
\textbf{Feature} & \textbf{Cellebrite UFED} & \textbf{Magnet AXIOM} & \textbf{Autopsy/TSK} \\
\hline
Extraction & Logical, filesystem, physical; no-root bypass & Express via Graykey (35-45 min) & Logical carving; root often required \\
\hline
Geolocation & Timeline mapping, LTE CSFB recovery \cite{Sutikno2024} & AI-triaged paths \cite{Magnetforensics2025} & EXIF/polyline decoding \cite{Sutikno2024} \\
\hline
Usability & Polished GUI, auto-reports & Streamlined multi-workstation & Dense interface, steep CLI curve \cite{Cybervie2021} \\
\hline
Cost (BE 2025) & €75k/year license; 2-3 shared & Subscription model & Free \\
\hline
Charleroi Use & Overflow cases despite backlog & 95\% success Samsung/iPhone & Never used \\
\hline
\end{tabular}
\caption{Comparison of forensic tools for mobile geolocation analysis.}
\label{tab:tools}
\end{table}
\subsection{Summary forensics and digital investigation}
\label{secsotaforensicssummary}

Forensic work on smartphone location traces focuses on how traces are generated, how long they persist, and how they can be recovered and analysed. Error classification separates systematic, random, and gross errors, with systematic errors dominating in digital forensics because of tool design and standardised procedures \cite{Nelson2015}. The three phases of production, persistence, and investigation introduce distinct points where accuracy can degrade and therefore require explicit evaluation.

Location traces form essential digital evidence and arise from several sources. Dedicated positioning technologies generate explicit location records, while EXIF metadata, communication records, and browser history provide incidental but usable positioning information. Each source type has its own reliability profile, shaped by the acquisition process and how long the data remain stored.

Strong encryption on Android (FDE/FBE) and iOS (filesystem-level encryption) constrains direct access to device contents. Commercial forensic platforms such as Cellebrite UFED and Magnet AXIOM offer extensive extraction functions with validated workflows, but their licensing costs restrict availability to some agencies. Open-source tools like Autopsy and ALEAPP reduce financial barriers but provide fewer specialised features, depend on community support, and often require root access. Successful investigations rely on choosing suitable tools, validating them regularly, and understanding the technical mechanisms that create and store traces.

\section{Network and positioning technologies}
\label{secnetworkpositioning}

\subsection{Different types of positioning technologies}
\label{secsotalocationtracetypes}

Positioning technologies generate location traces with specific accuracy and reliability patterns.

\textbf{Global Navigation Satellite Systems (GNSS):} GNSS receivers including GPS calculate position through satellite trilateration. Smartphones achieve positioning accuracy within a few meters under clear sky conditions \cite{Carney2025}. GNSS traces record latitude, longitude, elevation, speed, timestamps, and accuracy estimates that indicate signal quality \cite{Carney2025}. Direct line-of-sight to satellites proves essential; indoor and urban canyon environments severely attenuate signals.

\begin{figure}[H]
    \centering
    \includegraphics[width=0.60\textwidth]{gps_trilateration.png}
    \caption{GPS trilateration using satellite signals.\cite{olson2018gps}}
    \label{fig:gps_trilateration}
\end{figure}

\textbf{Cellular networks:} Mobile phones generate Call Detail Records (CDRs) when connecting to cell towers. Location accuracy ranges from kilometers in rural areas to hundreds of meters in cities \cite{Oikonomidis2019}. Tower selection depends on frequency, density, power, obstacles, congestion, and weather \cite{Oikonomidis2019}. Sector coverage creates large uncertainty areas due to overlapping boundaries and handoffs between towers \cite{Mediaforensics2023}.

\begin{figure}[H]
    \centering
    \includegraphics[width=0.75\textwidth]{triangulation.png}
    \caption{Cellular network triangulation for location estimation. A mobile device connects to multiple cell towers (Cell ID 1, 2, and 3) with varying signal strengths (RX1, RX2, RX3). The serving cell and signal measurements from adjacent cells enable triangulated location calculation \cite{damonam2012types}.}
    \label{fig:cellular_triangulation}
\end{figure}

\textbf{WiFi positioning:} Devices measure WiFi signal strength from access points listed in crowd-sourced databases. Accuracy reaches 5-10 meters in good conditions \cite{Grow-space2025}. Database errors and directional biases toward populated areas limit reliability \cite{Spichiger2023}.

\begin{figure}[H]
    \centering
    \includegraphics[width=0.75\textwidth]{WIFI-Diagram04-RSSI-Fingerprinting.png}
    \caption{Wi-Fi RSSI fingerprinting for indoor positioning. \cite{inpixon2024wifi}.}
    \label{fig:wifi_rssi_fingerprinting}
\end{figure}

\textbf{Bluetooth Low Energy (BLE):} BLE beacons support RSSI-based fingerprinting with 1-5 meter accuracy \cite{Grow-space2025}. Bluetooth 5.1 adds Angle of Arrival for sub-meter precision \cite{Inpixon2024}.

\textbf{Ultra-Wideband (UWB):} UWB measures precise time-of-flight distances with 10-30 cm accuracy indoors \cite{Grow-space2025}.

\textbf{Combined methods:} Devices integrate multiple positioning sources. Cellular or WiFi provides coarse estimates. GNSS refines positions when available. Sensors maintain tracking during signal loss.

\subsection{Cellular network geolocation measurement framework and influencing factors}
\label{seccellulargeolocation}

Cellular network signals complement satellite positioning in smartphone geolocation analysis. Signal propagation models, geometric constraints, and environmental effects determine the quality of cellular-based position solutions.


\subsection{Summary network and positioning technologies}
\label{secsotapositioningsummary}

Cellular positioning determines device location by analyzing measurements from mobile network base stations to which the device connects. Unlike Global Navigation Satellite Systems (GNSS) that rely on satellite signals receivable only with clear sky visibility, cellular positioning leverages the wireless infrastructure already deployed globally for voice and data communications. This approach proves valuable in urban environments where GNSS signals are obstructed by buildings, in indoor environments where satellite signals are severely attenuated, and in forensic applications where multiple positioning modalities provide cross-validation.

Cellular positioning operates through two primary mechanisms: explicit positioning services provided by network operators, and implicit positioning data generated during normal device-network communication \cite{CableFree2018}. Explicit mechanisms including Assisted GPS and network-based positioning services have been documented extensively. This master thesis focuses on implicit positioning information embedded in cellular signal measurements that devices must report to network infrastructure for cell selection, handover decisions, and signal quality assessment.

\subsubsection{Signal measurement parameters}

LTE devices measure signal characteristics from base station reference signals \cite{CableFree2018, ArImas2016}. Three main parameters support geolocation analysis.

\textbf{Reference Signal Received Power (RSRP).} RSRP measures received power from one base station sector. Values appear in dBm units. The measure averages power across reference signal resource elements \cite{ArImas2016}. Readings range from -75 dBm close to towers to -120 dBm at coverage edges \cite{CableFree2018}. RSRP focuses on sector-specific power. It excludes interference from nearby sectors. This focus improves distance estimation compared to broadband measurements.

\textbf{Reference Signal Received Quality (RSRQ).} RSRQ measures signal quality against noise and interference \cite{CableFree2018, ArImas2016}. The formula calculates:

\[
\text{RSRQ} = \frac{N \times \text{RSRP}}{\text{RSSI}}
\]

$N$ represents the number of physical resource blocks in the measurement bandwidth. RSRQ values span -14 dB to -8 dB in working networks. High values near -8 dB show clean conditions. Low values near -14 dB show noise or interference. Poor RSRQ increases uncertainty in distance calculations for positioning.

\textbf{Received Signal Strength Indicator (RSSI).} RSSI measures total received power across the full bandwidth \cite{CableFree2018}. The total includes serving cell signals, neighbor cell interference, and noise:

\[
\text{RSSI} = 12 \times N \times \text{RSRP}
\]

This equation applies during full resource block activity and high signal-to-noise ratio \cite{CableFree2018}. RSSI covers too broad an area for precise distance estimation. Still, RSSI compared to RSRP reveals multipath and non-line-of-sight conditions.

\textbf{Timing Advance (TA).} TA measures roundtrip delay between base station and device \cite{CableFree2018}. This delay gives direct distance estimates. No path loss model calibration is needed. TA resolution spans meters to tens of meters per step. TA pairs effectively with RSRP data to improve trilateration solutions.


\subsubsection{Distance estimation: Path Loss propagation models}

Signal strength converts to distance through propagation models \cite{Kurner2017}. Free space signals decay with $d^{-2}$ where $d$ is distance. Real environments add obstacles. The path loss equation captures this:

\[
\text{Path Loss (dB)} = A + 10n \log_{10}(d)
\]

$A$ sets reference loss at 1 meter. $n$ is the path loss exponent. $d$ measures distance in meters \cite{Kurner2017, Vo2024}. Distance solves as:

\[
d = 10^{(A - \text{RSRP})/(10n)}
\]

RSRP uses dBm units \cite{Vo2024}. The exponent $n$ reflects environment type. Free space uses $n = 2$. Urban areas use $n = 3.5$ to $5.5$ \cite{Kurner2017}. Dense urban canyons reach highest values.

Frequency affects propagation. Lower bands at 700 MHz show $n \approx 3.5$-$4.0$ \cite{Kurner2017, Vo2024}. Higher bands at 2.6 GHz show $n \approx 5.0$-$5.7$ \cite{Kurner2017, Vo2024}. Separate calibration per frequency band increases accuracy.

\subsubsection{Bearing and directional information}

Base stations deploy three-sector antennas covering 120° each \cite{CableFree2018}. Antenna azimuth spans 0° to 360°. This marks each sector's main beam direction. Main lobe signals yield high RSRQ. Side lobe signals show lower RSRQ. Devices align near the antenna azimuth \cite{CableFree2018}.

Serving cells provide bearing estimates with ±30-60° uncertainty \cite{CableFree2018}. Neighbor cells give mainly distance data. Their side lobe signals degrade RSRQ heavily. Bearing uncertainty reaches ±90° or more.

Multiple cells combine distance and bearing data. Distance creates circles around towers. Bearing adds directional wedges. Both factors with confidence levels refine position estimates.

\subsubsection{Geometric quality: Dilution of precision framework}

Geometric Dilution of Precision (GDOP) measures base station geometry impact \cite{Wikipedia2004}. Same measurement accuracy yields different errors by tower layout. GDOP acts as error amplification factor:

\[
\text{GDOP} = \frac{\Delta(\text{position})}{\Delta(\text{measurement})}
\]

GDOP = 1 shows optimal geometry \cite{Wikipedia2004}. Values of 5-10 work well. Values of 10-20 perform adequately. Values above 20 degrade results.

Towers surrounding devices at 120° angles minimize GDOP \cite{Ramadhani2020}. Collinear towers with devices maximize GDOP. Urban streets cluster towers linearly. Sparse areas spread towers widely. Device movement changes geometry over time \cite{Ramadhani2020}.

\subsubsection{Signal degradation mechanisms: Multipath and NLOS}

Multiple signal paths cause multipath. Direct paths combine with reflections \cite{NCBI2024}. Reflections delay arrival. Delayed signals distort direct signals. RSRP, RSRQ, and distance estimates shift.

RSSI compared to RSRP detects multipath. Large RSSI-RSRP differences show interference \cite{NCBI2024}. RSRQ below expected values confirms multipath presence.

Obstacles block direct paths in NLOS conditions. Signals diffract or reflect around barriers. Path length increases. Attenuation grows. Path loss models fail without line-of-sight assumptions.

NLOS reduces RSRQ at equal power levels \cite{NCBI2024}. RSRP mismatches distance predictions. RSSI-RSRP spread plus RSRQ drop signals NLOS conditions.

\subsubsection{Trilateration: Geometric position computation}

\textbf{Distance circle intersection}

Each distance measurement defines a circle of possible positions. Three non-collinear circles intersect at one point, the estimated position. More measurements produce overdetermined systems. Algorithms select the point minimizing residual error across all circles.

Bearing information adds wedge-shaped constraints. Combined distance and bearing narrower the solution space compared to distance alone.

\textbf{Residual error and solution optimization}

Overdetermined systems (four+ measurements) require optimization. Least-squares algorithms minimize total residual error weighted by measurement confidence. This produces position estimates with associated residuals and quality metrics.


\subsection{Summary: Network and positioning technologies}
\label{sec:sota_positioning_summary}

Positioning technologies operate on different principles. GNSS trilaterates satellites. Cellular networks use base station geometry. WiFi draws from access point databases. BLE fingerprints signal strength. UWB calculates time-of-flight \cite{Carney2025, Oikonomidis2019, Grow-space2025, Inpixon2024}.

Accuracy spans wide ranges by environment:
\begin{itemize}
\item GNSS hits meter-scale in clear sky. Indoor failure follows.
\item Cellular spans 200-800 m urban. Kilometers mark rural limits.
\item WiFi delivers 5-10 m reliably.
\item BLE achieves 1-5 m precision.
\item UWB reaches 10-30 cm indoors.
\end{itemize}

Environment limits methods differently. GNSS requires satellite line-of-sight. Cellular positioning encounters sector overlap and multipath interference. WiFi accuracy relies on complete databases. Multipath affects all technologies.

Frequency influences propagation significantly. Lower bands penetrate obstacles effectively. Path loss exponents range from 3.5 to 5.7 across frequency bands and terrain types \cite{Kurner2017, Vo2024}.

Hybrid positioning enhances reliability. GNSS excels outdoors. Cellular networks function indoors. WiFi provides supplementary data. Sensor fusion selects optimal signals dynamically.

Geometry amplifies errors via GDOP metrics \cite{Wikipedia2004, Ramadhani2020}. GDOP values of 1 indicate optimal tower arrangement. Scores from 5-10 support good performance. Values above 20 cause substantial degradation.

Multipath combines direct signals with reflections from structures. NLOS propagation routes signals around obstacles, extending path lengths \cite{NCBI2024}. RSSI-RSRP differences reveal these impairments.

Trilateration weights signals by RSRQ quality. Strong signals influence position calculations more heavily. Degraded signals contribute minimally.

GDOP below 10 paired with clean signals produces 50-150 meter uncertainty. Poor geometry reaches 200-800 meter errors \cite{CableFree2018}.

Forensic work demands awareness of these limits. Analysts quantify environmental effects precisely. Uncertainty reporting maintains evidence integrity.


\chapter{Experimental Framework and Device Selection Journey}
\label{ch:experimental_framework}

\section{Overview: From Cellebrite to LineageOS}
\label{sec:exp_overview}

Initial plans targeted commercial forensic tools like Cellebrite. Ethical constraints required open-source alternatives. Multiple devices underwent testing, each encountering distinct technical barriers: encryption protection, firmware unavailability, operating system fragmentation, and bootloader restrictions.

The final setup combines three data sources. G-NetTrack Pro logs continuous GNSS and network signals. Autopsy extracts discrete EXIF metadata from photos. Garmin Venu Sq 2 watch provides independent reference positioning.

This approach delivers synchronized smartphone, network, and ground truth data across all test environments.

\section{Part 1: Initial plan and ethical considerations}
\label{sec:exp_ethics}

\subsection{Original methodology}

The initial research plan involved extracting geolocation traces from smartphones using Cellebrite UFED, a leading commercial forensic platform, then analyzing them in the laboratory at Charleroi's Police Judiciaire Fédérale. This approach would have replicated real forensic workflows used by law enforcement and provided data extraction methods closest to operational reality.

\subsection{Ethical constraint}

Charleroi lab had three active Cellebrite licenses. These ran nonstop on criminal cases. Academic use would divert resources from police operations and create unacceptable opportunity costs.

Prioritizing personal research over active investigations violated ethics. The decision stood clear: develop alternative extraction methods that avoid competing with public safety work.

\section{Part 2: Autopsy framework transition}
\label{sec:exp_autopsy}

\subsection{Rationale for change}

Autopsy (version 4.22.1) provided a viable open-source alternative. Unlike Cellebrite, it required no licensing fees and ran independently of lab infrastructure.

\subsection{Analysis workflow}

The workflow used three steps: USB connection, image extraction, and metadata parsing.

Devices connected via USB using Android Debug Bridge (adb). ADB pull command extracted images from phone memory to workstation. Autopsy then loaded the phone image data. ALEAPP module within Autopsy parsed Camera Photos EXIF data, including GPS coordinates captured at checkpoints, timestamps for alignment, and altitude values.

\subsection{Autopsy capabilities and limitations}
\label{sec:autopsy_limits}

Autopsy 4.22.1 uses aLEAPP (Android) and iLEAPP (iOS) ingest modules. 
These modules parse artifacts from phone storage images. 
They process connected accounts, SMS/MMS messages, call logs, WiFi connections, Bluetooth pairings, browser history, installed applications, and selected app-specific databases.\cite{aleapp_doc,ileapp_doc}

This case study identified geolocation evidence only from EXIF metadata in photographs. 
Autopsy 4.22.1 extracted EXIF geolocation data and basic device configuration. 
The tool failed to recover continuous GPS tracks, network logs, cell tower signals, or detailed network technology data.

aLEAPP could not extract encrypted location search history from applications. 
This limitation blocked access to traces not designed for geolocation. 
Examples include browser history, application caches, search records, and communication artifacts.

Autopsy provided only discrete image metadata from checkpoints. 
It supplemented continuous positioning data collection but did not replace it.

\section{Part 3: Device selection journey}
\label{sec:exp_device_journey}

\subsection{Samsung Galaxy A51 - Encryption barrier}
\label{subsec:device_samsung}

\textbf{Device specifications.} Samsung Galaxy A51 running Android 13, last security patch from 2022 (outdated). The device rooted successfully using standard tools.

\textbf{Challenge encountered.} Samsung implements File-Based Encryption (FBE) on top of Android's standard encryption framework. The firmware includes vbmeta (verified boot metadata) that cryptographically prevents access to encrypted user data partitions, even when the device is physically rooted.

Root access alone proved insufficient. Bootloader verification validates system and boot partition integrity before permitting data access. Without disabling vbmeta (which requires bootloader unlocking), encrypted data remained inaccessible. Device metrics and configuration files were readable; the actual geolocation database contents remained encrypted.

\textbf{Decision.} Samsung's protective measures proved too robust for available tools and time constraints. A different manufacturer's device became necessary.

\subsection{OnePlus Nord2 5G - Firmware availability problem}
\label{subsec:device_oneplus_nord}

\textbf{Device specifications.} OnePlus Nord2 5G originally shipped with Android 12. Security patches current as of November 5, 2024. Bootloader access was denied.

\textbf{Rationale for selection.} OnePlus marketed the Nord2 5G as a "customizable and accessible phone" with explicit commitments to user-friendly modifications and open firmware access. This positioning made it an attractive candidate for research requiring root access and custom configurations.

\textbf{Problem encountered.} OnePlus's public firmware availability declined progressively. For the Nord2 5G, official firmware downloads were either restricted or no longer available through standard channels. Additionally, Android 13 introduced significantly hardened bootloader security measures across the Android ecosystem, making bootloader unlock attempts increasingly difficult.

An alternative approach attempted downgrading the device to Android 11, where bootloader access was reportedly less restrictive. However, older firmware versions required for downgrading no longer existed on OnePlus servers. Without firmware files, downgrade flashing became infeasible.

\textbf{Contributing factors.} Removal of historical firmware repositories by the manufacturer, absence of third-party firmware hosting options with verified authenticity, and time constraints discouraging searches for leaked or mirrored firmware files.

\textbf{Decision.} Abandon the Nord2 approach. A device with publicly available recent firmware and active community support became essential.

\subsection{Huawei P8 - Android fragmentation issue}
\label{subsec:device_huawei}

\textbf{Device specifications.} Huawei P8 Lite 2017 running Android 7 (Nougat), last security patch from May 1, 2018. Rooting proved straightforward using standard tools.

\textbf{Challenge encountered.} The Android version gap created insurmountable methodological problems. Between Android 7 (2016) and Android 13 (2022), the operating system underwent multiple architectural revisions \cite{Android2024VersionHistory}.

Android 8 (Oreo) introduced foreground versus background process distinction, limiting background location access for power efficiency \cite{HyperTrack2020, Android2024Oreo}. Android 13 implemented granular geolocation permission controls and decoupled WiFi and Bluetooth scanning from location permissions, fundamentally changing how apps access positioning data \cite{Android2023Behavior, GeeksForGeeks2023, Esper2022}.

Data extracted from Android 7 would represent geolocation behavior from nearly seven years prior and would not accurately reflect contemporary smartphone behavior. The experimental goal was understanding current geolocation quality, not historical trends.

\textbf{Decision.} Device too outdated to provide meaningful contemporary data.

\subsection{OnePlus 6T - Successful implementation}
\label{subsec:device_oneplus_6t}

\textbf{Device specifications.} OnePlus 6T originally shipped with Android 13. Upgraded to LineageOS 22.2 based on Android 15. Security patch current as of November 2025 (latest available). Rooted successfully using Magisk v29.0. Bootloader unlocked after considerable technical effort.

\textbf{Success factors.} Three factors enabled successful implementation. First, the OnePlus 6T benefits from large and active development community. Comprehensive rooting guides, custom ROM documentation, and firmware mirrors are readily available. Second, bootloader unlocking, while requiring technical knowledge and persistence, had community resources providing detailed procedures that proved effective. Third, custom ROM (LineageOS 22.2) is actively maintained and receives current security updates.

\textbf{LineageOS 22.2 configuration.} LineageOS 22.2 was selected as the base operating system. Modularity allows granular configurability of location services. Transparency through open-source codebase enables verification of data handling practices. Minimal bloatware reduces background processes that might interfere with controlled measurements. Enhanced granularity in location permission management improves privacy controls.

\textbf{Location service configuration.} To isolate the impact of different geolocation methods, the following were disabled: WiFi scanning (preventing WiFi access point trilateration) and Bluetooth scanning (preventing BLE beacon-based localization).

Remaining active location sources: GPS (Global Positioning System) and cellular network location (cell tower triangulation and 4G/LTE positioning). This configuration ensures measurement of only GPS and network-based localization, allowing direct comparison between satellite-based and cellular-based positioning accuracy.

\textbf{Location permissions.} G-NetTrack Pro received the following permissions: notification access, file access, and location access while the application is active.

\textbf{Development environment.} USB debugging was enabled solely for application installation. No special developer mode was activated. Device operated in standard user configuration.
\begin{figure}[H]
    \centering
    \hspace*{-2cm}
    \includegraphics[width=1.2\textwidth]{figures/workflow-phone.png}
    \caption{Complete device selection workflow for mobile forensics analysis.}
    \label{fig:device-workflow}
\end{figure}
\newpage
\section{Final Configuration}
\label{sec:exp_final_config}

\subsection{Hardware setup}

The production environment integrated four components. 

OnePlus 6T smartphone ran LineageOS 22.2 (Android 15, November 2025 patch level). 
The phone used Magisk v29.0 for rooting. 

G-NetTrack Pro v3.75 logged continuous cellular and GPS data on the phone. 

Garmin Venu Sq 2 smartwatch provided independent ground truth positioning. 

Autopsy 4.22.1 workstation extracted EXIF metadata from images.

\subsection{Data collection tools}

\textbf{G-NetTrack Pro v3.75.} Primary tool for continuous geolocation and cellular measurements. Configured with 1-second intervals, verbose mode, cell info logging, IMSI/IMEI reporting, and network technology recording. Installed via external APK.

Each session produces tab-delimited logs with 200-2000 entries. Fields include timestamp (ms precision), latitude, longitude, altitude, speed, bearing, accuracy, serving cell (CID, LAC, CGI), operator, network type (LTE/GSM), RSRP, RSRQ, RSSI, timing advance, and up to 18 neighbor cells per record.


\textbf{Garmin Venu Sq 2.} Consumer-grade GNSS receiver serving as independent ground truth reference. Garmin GPS receivers are accurate to within 15 meters 95\% of the time under clear sky conditions \cite{Garmin2024}. Typical accuracy ranges from 3-10 meters \cite{Garmin2024}. Venu Sq 2 supports single-band GPS with GLONASS and Galileo constellation support \cite{The5KRunner2022}.  Configured to record GPS data at 6-second intervals.

\textbf{Autopsy 4.22.1.} EXIF metadata extraction from checkpoint photographs. Extracted image files via adb from device storage, parsed GPS coordinates, timestamps, altitude, and confidence metrics from EXIF metadata.

\textbf{Custom data processing scripts.} Python pipeline processes raw logs through three main stages.

\begin{itemize}[noitemsep]
\item \textbf{Network analysis:} Parses logs, verifies coverage, extracts signals, calculates distances, computes bearings, trilaterates, validates results.
\item \textbf{GPS comparison:} Parses Garmin data, aligns trajectories, flags quality issues, merges datasets, computes errors, analyzes cloud effects, creates visualizations.
\item \textbf{Distance regression:} Pairs measurements with towers, trains models, selects best performer, exports formulas.
\end{itemize}

Code: \url{https://github.com/romain-mottet/forensic-gnss-lte-analysis}

\section{Key lessons for replication}
\label{sec:exp_lessons}

\subsection{Device selection criteria}

Researchers attempting to replicate this work should prioritize several device characteristics.

\begin{itemize}[noitemsep]
\item \textbf{Active community support.} OnePlus, Motorola, Xiaomi devices have better documentation and firmware availability than niche manufacturers. Forums and rooting guides cut implementation time substantially.
\item \textbf{Recent rooting tools.} Android rooting changes rapidly with new manufacturer security measures. Check Magisk compatibility and community forums before selection.
\item \textbf{Operating system currency.} Pick devices with latest Android versions through official updates or custom ROMs. Devices over 4-5 years old lack modern behavior.
\item \textbf{Firmware accessibility.} Verify firmware files on manufacturer servers and XDA Forums before purchase. Samsung and recent iPhones create encryption barriers.
\item \textbf{Bootloader status.} Confirm bootloader unlock availability. OnePlus and Motorola permit unlocks. Apple and Samsung enforce permanent restrictions.
\end{itemize}

\subsection{Data collection considerations}

6-second logging generates large datasets. Scripts handle parsing and timestamp alignment across sources. Initial stand-move-stand sequence synchronizes Garmin watch with phone logs reliably.

Discrete checkpoint photos validate continuous data. EXIF metadata confirms accuracy at specific moments. Continuous logs reveal dynamic movement patterns.

\subsection{Ground truth reference}

Garmin Venu Sq 2 delivers 5-15 meter accuracy. Consumer GNSS receivers like this work well for relative smartphone comparisons. Consistency matters more than absolute position here.

Professional surveying receivers hit sub-meter precision. Use these for true accuracy tests.

Ground truth choice sets error analysis limits. Consumer gear compares smartphones fine. Forensic work demands high-accuracy references for defensible results.
\chapter{Methodology}
\label{chap:methodology}

This chapter evaluates the framework from Chapter~3 against the research questions RQ1 and RQ2 defined in Section~\ref{sec:intro_research_questions}. 
RQ1 measures smartphone GNSS accuracy by comparing device coordinates with Garmin Venu Sq~2 references and EXIF-derived waypoints, while RQ2 validates LTE signal-based distance models used for trilateration in the absence of GPS. 

The analysis relies on 15 data-collection sessions conducted across 3 Belgian environments. 
Urban measurements took place in Ixelles (Brussels Capital Region), suburban routes followed predefined paths in Louvain-la-Neuve (Brabant Wallon), and rural tracks were recorded around Waha (Luxembourg Province). 
Each $\approx$ 800~m walk included five fixed waypoints: start, photo1, photo2, photo3, and end, ensuring repeated observations at the same physical locations. 

Sessions occurred on different dates and at varied times of day in order to capture changes in satellite geometry, weather conditions, and network behaviour.

\section{Experimental setup}

The test configuration follows the final device selection from Section~\ref{sec:exp_device_journey}. Forensic reproducibility guides this setup.

\begin{itemize}
\item \textbf{Smartphone:} OnePlus 6T runs LineageOS 22.2 (Android 15). Magisk v29 provides root access. G-NetTrack Pro v0.375 logs networks at 6~s intervals. Logs include IMSI/IMEI/MSISDN and full cell data in verbose mode. Camera photos embed EXIF GPS metadata. These serve as primary forensic ground truth.
\item \textbf{Smartwatch:} Garmin Venu Sq 2 delivers independent GNSS reference. GPX track export enables cross-validation. Smartphone Android location APIs stay excluded.

\item \textbf{Forensic workstation:} ADB root extraction produces tar.gz filesystem archives. Autopsy 4.22.1 processes these with aLEAPP. Output creates ground\_truth\_waypoints.csv. This file holds EXIF coordinates, timestamps, and session metadata tied to physical locations.
\end{itemize}

A strict protocol prepares devices before each series. Factory reset clears persistent state. Disabled WiFi/Bluetooth scanning blocks assisted positioning artifacts. Manual time sync aligns timestamps across devices.

See Annex~\ref{annex:device-prep} for full preparation details.

\section{Field data collection protocol}
Each session follows a standardized procedure ensuring comparability across environmental conditions and repetitions:

\begin{enumerate}
    \item \textbf{Initialization:} Confirm GPS lock accuracy $\leq$10~m, launch G-NetTrack Pro logging, start Garmin activity recording, and document cloud coverage on 0--5 scale (0=clear sky, 5=overcast) as primary environmental covariate.
    \item \textbf{Route execution:} Traverse waypoints in fixed sequence taking timestamped photographs (EXIF ground truth); between photo2 and photo3, activate Google Maps navigation to a nearby shop simulating realistic foreground application usage that influences background location behavior.
    \item \textbf{Repetition:} Conduct five sessions per environment on different days/times capturing diurnal satellite geometry variations and network load differences.
\end{enumerate}

This protocol generates synchronized multimodal datasets: raw LTE/GPS logs, Garmin GPX tracks, forensic EXIF waypoints, and environmental metadata enabling comprehensive geolocation quality assessment.

\section{Analysis pipeline}
Figure~\ref{fig:diagram} shows the three-stage workflow. Raw field data transforms into forensic positioning metrics. Custom scripts from Chapter 3 process EXIF waypoints, GPS comparisons, LOFO regression, and LTE trilateration.

\begin{figure}[htbp]
    \centering
    \hspace*{-2cm}
    \includegraphics[width=1.2\textwidth]{figures/diagram.pdf}
     \caption{Three-stage pipeline: field collection → forensic extraction → Different python script analysis.}
    \label{fig:diagram}
\end{figure}

\subsection{Stage 1: GPS comparison pipeline (12 Steps)}

The GPS analysis pipeline executes twelve sequential steps to validate and compare smartphone and Garmin watch GPS tracks against ground-truth waypoints. These steps convert raw data, align timestamps optimally, flag quality issues, compute accuracy metrics, and generate visualizations.

Parameters such as time windows (0.5--8.0~s), offsets ($\leq$3600~s), gaps ($\geq$12~s unstable, $\geq$60~s major), waypoint matching (25~s), and phone-watch agreement (10~m) are tuned for this experiment but can be adjusted based on context and desired outcomes. The pipeline code is modular, with all key parameters centralized in a dedicated configuration file (src/config/parameters.py) that allows easy modifications without altering the core implementation.

\begin{enumerate}
    \item \textbf{GPX to CSV conversion}: Recursively locate all Garmin watch .gpx files and convert them to CSV format (gps\_groundtruth.csv), preserving timestamps, coordinates, and accuracy fields like HDOP where available.
   
    \item \textbf{Optimal time window detection}: For each smartphone-watch pair, test alignment windows from 0.5 to 8.0 seconds in 0.5-second increments to identify the window maximizing match density.
   
    \item \textbf{Smartphone-watch merging}: Merge datasets using the optimal window and correct clock offsets up to 3600 seconds (1 hour) to handle DST changes; reject pairs exceeding this limit. Output per-session comparison files.
   
    \item \textbf{Quality gap analysis}: Scan merged files for duplicate timestamps and gaps, flagging unstable gaps $\geq$12 seconds and major gaps $\geq$60 seconds. Generate quality reports.
   
    \item \textbf{Data flagging}: Add quality flags to comparison data, including ``iscacheduplicate'' for duplicates and ``nearmajorgap30s'' for records within 30 seconds of major gaps.
   
    \item \textbf{Unified dataset creation}: Concatenate flagged files across sessions, enrich with metadata like cloud coverage from waypoints, and produce unified\_gps\_dataset.csv.
   
    \item \textbf{Waypoint distance matching}: Match GPS fixes to ground-truth waypoints within 25-second temporal windows, computing Haversine distances between coordinates. Generate per-session accuracy reports.
   
    \item \textbf{Global device accuracy}: Aggregate waypoint distances to compute per-device metrics: CEP (50th percentile), R95 (95th percentile), and RMSE.
   
    \item \textbf{Accuracy by location}: Stratify accuracy metrics (CEP, R95, RMSE) by location for environmental context.
   
    \item \textbf{Accuracy by cloud coverage}: Group metrics by cloud coverage categories (0=clear to 5=overcast) to assess environmental impact.
   
    \item \textbf{Phone-Watch agreement}: At aligned timestamps, flag instances where both devices report positions within 10 meters of each other.
   
    \item \textbf{Visualization generation}: Produce six PDF figures, including accuracy heatmaps, scatter plots, boxplots, and trends by location, device, and cloud coverage.
\end{enumerate}

\subsection{Stage 2: LTE distance regression}
\label{stage:lte-distance-regression}

This pipeline estimates LTE cell tower distances from signal measurements (RSRP, RSRQ, EARFCN) using parsed logs from stage 3. It associates measurements with ground-truth tower locations, engineers features, performs Leave-One-File-Out (LOFO) cross-validation, selects optimal models, and exports formulas in json formula directly usable in stage 3.

\begin{enumerate}
    \item \textbf{Input data ingestion}: Load parsed LTE measurement CSV files (parsed\_*.csv) containing timestamps, GPS coordinates, RSRP, RSRQ, PCI, and EARFCN for serving/neighbor cells.
   
    \item \textbf{Tower database joining}: Match measurements to tower locations using primary key (PCI + EARFCN) from towers.json; fallback to PCI-only matching via pci.json if enabled. Flag and exclude unmatched entries.
   
    \item \textbf{Ground-truth distance computation}: Calculate Haversine distances \(d_{\text{true}}\) (in meters) between receiver GPS and matched tower coordinates.
   
    \item \textbf{Log-space target framing}: Transform distances to \(y = 10 \log_{10}(d_{\text{true}})\) to stabilize variance, enforce positivity, and align with log-distance path loss models.
   
    \item \textbf{Feature engineering}: Prepare linear features (RSRP, RSRQ, EARFCN$_k$/1000) or nonlinear expansions (squares: rsrpsq/rsrqsq; interaction: rsrpxearfcnk). Limit to max 18 neighbors (limit from G-NetTrack Pro), per timestamp, filtering non-LTE.
   
    \item \textbf{LOFO Cross-Validation}: Perform Leave-One-File-Out validation rotating across measurement files per environment/context to test generalization.
   
    \item \textbf{Model evaluation and selection}: Fit linear regression models per fold; select the best minimizing MAE, RMSE, and MAPE across baseline and nonlinear variants. Log detailed per-fold metrics.
   
    \item \textbf{Formula export}: Save optimal model as formula\_*.json with coefficients (e.g., intercept 0.9427, coefrsrp -0.00038), features, and notes for downstream use.
\end{enumerate}


\subsection{Stage 3: LTE network positioning (9 Phases)}

The LTE positioning pipeline runs nine phases. Four environmental contexts guide it: \textbf{default}, \textbf{town}, \textbf{city}, \textbf{village}. See Section~\ref{subsec:env_contexts}. Stage 2 formula variants feed Phase 2.2 for distance estimates. These enable model comparisons. Outputs match Haversine distances and GPS ground truth. Path-loss random errors beat formula precision in trilateration. Bias explains this.

\subsubsection{Environment contexts: assumptions and parameter calibration}
\label{subsec:env_contexts}

Propagation environment shapes LTE accuracy. Signal geometry matters too. Four contexts handle this: \textbf{default}, \textbf{city}, \textbf{town}, \textbf{village}. Each sets parameters for signal traits and cell density. Contexts set Phase 2.2 distance uncertainty. They also control Phase 3.1--3.2 trilateration limits.

\paragraph{Path Loss exponent and distance estimation}

All contexts use this path-loss model:
\begin{equation}
d = 10^{\frac{\text{RSRP}{\text{ref}} - \text{RSRP}{\text{meas}}}{10n} + h(\text{RSRQ})}
\end{equation}

$n$ is the path-loss exponent. It spans 2.7-2.82 for LTE bands 700--2600 MHz. ITU-R M.1225 and 3GPP TS 36.942 provide data~\cite{ITU-R:2008, 3GPP:2023}. Exponents stay fixed across contexts. Contexts adjust distance uncertainty bounds instead. RSRP quality sets these bounds. Band physics reuse holds. Environmental signal stability varies.

\paragraph{Signal quality and uncertainty classification}

RSRP falls into six levels. Thresholds apply everywhere:
\begin{itemize}
\item \textbf{EXCELLENT} ($\text{RSRP} > -85 \text{ dBm}$): Near-tower. Fading variance $\pm$2--5 dB.
\item \textbf{HIGH} ($-85$ to $-100$ dBm): Moderate multipath. Fading variance $\pm$5--10 dB.
\item \textbf{GOOD} ($-100$ to $-110$ dBm): Non-line-of-sight. Fading variance $\pm$8--12 dB.
\item \textbf{MEDIUM} ($-110$ to $-120$ dBm): Multipath and shadowing. Fading variance $\pm$15--20 dB.
\item \textbf{WEAK} ($-120$ to $-130$ dBm): Coverage edge. Variance unpredictable.
\item \textbf{VERY_WEAK} ($< -130$ dBm): Extreme edge. Estimates fail.
\end{itemize}

Max uncertainty in meters shifts by context. Cities use 80-250 m. Dense towers help. Villages use 100-400 m. Sparse coverage demands this. Towns take middle values.

\paragraph{City context}

High tower density overlaps coverage. Strong signals select easily. Max cells $= 8$. $\text{GDOP}_{\text{max}} = 8.0$.
\begin{itemize}
\item $> 6$ cells visible. Pick 4-8 strongest.
\item Urban canyons raise $n \approx 2.8$. Matches 3GPP UMi.
\item Filter RSRP at $-110$ dBm.
\item GDOP $\leq 8.0$. Convergence 0.5 m.
\item Bounds cut 20 m from default. HIGH: 100 m.
\end{itemize}

\paragraph{Town context}

Moderate density mixes line-of-sight and non-line-of-sight. Max cells $= 6$. $\text{GDOP}_{\text{max}} = 9.0$.
\begin{itemize}
\item 3-6 cells visible.
\item Multipath moderate.
\item Filter RSRP at $-115$ dBm.
\item GDOP $\leq 9.0$. Convergence 0.75 m.
\item Bounds cut 10--20 m. HIGH: 110 m.
\end{itemize}

\paragraph{Village context}

Sparse coverage means large cells. Max cells $= 5$. $\text{GDOP}_{\text{max}} = 10.0$.
\begin{itemize}
\item 2--5 cells visible. Min cells $=2$.
\item Terrain raises $n \approx 3.5$-$4.0$. Fits 3GPP RMa.
\item Filter RSRP at $-120$ dBm.
\item GDOP $\leq 10.0$. Convergence 1.0 m.
\item Bounds match default. HIGH: 120 m.
\end{itemize}

\paragraph{Default context}

Baseline matches village. Max cells $= 5$. $\text{GDOP}_{\text{max}} = 10.0$. Use without context.

\paragraph{Bearing and trilateration weighting}

Solver weights by $w_i = 1/\sigma_i^2$. Phase 2.2 sets $\sigma_i$. This favors strong cells. GDOP filters bad geometry post-solution.

\paragraph{Parameters setup}

Files algorithm\_params.py and helpers\_config.py tune parameters. RSRP thresholds. Cell filters. GDOP $\leq8.0$. Weights. Changes avoid core code edits. Path-loss works. Stage 2 formulas ($d = 10^y$) fit. Batch and hybrid modes run.

\subsubsection{pipeline phases:}

\begin{enumerate}
\item \textbf{Phase 1.1: Parse LTE logs}: G-NetTrack .txt to parsed\_*.csv. Add timestamps, GPS, RSRP, RSRQ, PCI, EARFCN.

\item \textbf{Phase 1.2: Tower coverage verification}: Match to towers.json/pci.json. Flag unmatched cells in reports.

\item \textbf{Phase 2.1: Signal extraction}: Clean RSRP/RSRQ to signaldata.csv, metadata.csv. Apply thresholds.

\item \textbf{Phase 2.2: Distance estimation}: Path-loss (Friis) or Stage 2 ($d = 10^y$). Context-specific.

\item \textbf{Phase 2.2.5: Distance ground-truth validation}: Haversine check. MAE, RMSE, CEP, R95, bias. Report output.

\item \textbf{Phase 2.3: RSRQ-Weighted bearings}: Bearings from signals and distances.

\item \textbf{Phase 3.1: Trilateration preparation}: 4--8 cells. GDOP $\leq$8.0. RSRP thresholds.

\item \textbf{Phase 3.2: Weighted Least-Squares solving}: Minimize $\sum w_i (d_{\text{measured}} - d_{\text{model}})^2$. $w_i = 1/\sigma_i^2$.

\item \textbf{Phase 3.3: Position ground-truth validation}: GPS match. MAE, RMSE, CEP, R95, bias, GDOP. Output accuracysummary.txt.
\end{enumerate}

\section{Evaluation metrics}

Metrics quantify GNSS and LTE positioning accuracy, distinguishing systematic bias from random errors per Chapter 2 taxonomy.

\subsection{GNSS (Stage 1)}
Haversine distances measure errors from GPS fixes to EXIF-ground-truth waypoints.
- \textbf{CEP}: 50th percentile radius containing half of errors.
- \textbf{R95}: 95th percentile radius for extremes.
- \textbf{RMSE}: Root mean square error against Garmin watch references.
Phone-watch agreement flags co-located positions within 10~m.

\subsection{LTE distance (Stage 2 \& Phase 2.2/2.2.5)}
Regression models report errors in log-space predictions ($y = 10\log_{10}(d_{\text{true}}$)).
- \textbf{MAE}: Mean absolute error.
- \textbf{RMSE}: Root mean square error.
- \textbf{MAPE}: Mean absolute percentage error.
Phase 2.2.5 adds bias detection for overfitting (e.g., systematic 90~m underestimation).

\subsection{LTE position (Phase 3.3)}
Trilateration outputs validate against GPS ground truth.
- \textbf{CEP}/\textbf{R95}/\textbf{RMSE}: As above, for 2D positions.
- \textbf{Bias}: Mean signed error, highlighting directional offsets.
- \textbf{GDOP}: Geometric Dilution of Precision distributions (target $\leq$8.0).

\section{Quality assurance}

Automated controls ensure pipeline reliability across GNSS and LTE stages, with logging for reproducibility in default/town/city/village contexts.

- \textbf{Clock synchronization}: Correct offsets $\leq$3600~s (1 hour, DST-tolerant) via optimal windows (0.5--8.0~s).
- \textbf{Data cleaning}: Flag GPS gaps ($\geq$12~s unstable, $\geq$60~s major, 30~s proximity); tower coverage verification with discrepancy reports.
- \textbf{Overfitting prevention}: LOFO cross-validation in Stage 2 rotates across files.
- \textbf{Geometry filtering}: Trilateration rejects GDOP $>$8.0, limits 4--8 cells.
- \textbf{Distance validation}: Phase 2.2.5 benchmarks MAE/RMSE vs. Haversine truths.
- \textbf{Dependency logging}: Phase-by-phase outputs (e.g., parsed\_*.csv, distanceestimates.csv) enable modular debugging.
- \textbf{Config Reproducibility}: Centralized params.py/algorithm\_params.py for environment-specific tuning.

\chapter{Results}
\label{chap:results}

Fifteen sessions cover urban (Ixelles), suburban (Louvain-la-Neuve), and rural (Waha) sites. Results follow the three-stage pipeline. They validate RQ1 (GNSS via Garmin/EXIF) and RQ2 (LTE trilateration without GPS).

\section{GNSS accuracy (RQ1, Stage 1)}
\label{sec:gnss_results}

Stage 1 pipeline compares smartphone and Garmin watch against EXIF waypoints. Data spans 15 sessions (N=63 waypoints).

\begin{figure}[htbp]
\centering
\includegraphics[width=1\textwidth]{figure_02_device_comparison.pdf}
\caption{Device accuracy across sessions.}
\label{fig:gnss_devices}
\end{figure}

\subsection{Global device performance}

Smartphone beats Garmin across metrics (Table~\ref{tab:gnss_global}). CEP reaches 5.61 m versus 8.32 m. R95 stays at 14.79 m against 20.81 m. RMSE measures 9.47 m compared to 11.99 m. Std Dev hits 5.86 m versus 7.06 m.

\begin{table}[htbp]
\centering
\caption{Global accuracy (N=63 waypoints). From device\_comparison.csv.}
\begin{tabular}{|l|ccccc|}
\hline
Device & CEP & R95 & RMSE & Mean & Std Dev \\
\hline
Smartphone & 5.61 & 14.79 & 9.47 & 7.45 & 5.86 \\
Watch & 8.32 & 20.81 & 11.99 & 9.87 & 7.06 \\
\hline
\end{tabular}
\label{tab:gnss_global}
\end{table}

\subsection{Inter-Device agreement}

Both devices work at 866 timestamps (Table~\ref{tab:phone_watch}). They agree within 10 m at 68.0\% of cases. Median separation equals 7.7 m.

\begin{table}[htbp]
\centering
\caption{Phone-watch agreement (N=866 timestamps). From phone\_watch\_agreement.csv.}
\begin{tabular}{|l|r|}
\hline
Metric & Value \\
\hline
Agreement rate ($\leq$10 m) & 68.0\% \\
Mean distance & 8.9 m \\
Median distance & 7.7 m \\
95th percentile distance & 20.0 m \\
\hline
\end{tabular}
\label{tab:phone_watch}
\end{table}

\subsection{Environmental impact}

Environment drives accuracy. Waha (rural) leads with smartphone CEP at 3.59 m and RMSE at 6.34 m. Ixelles (urban) matches median at CEP 5.62 m. Outliers jump to R95 29.11 m. LLN (suburban) sits between at CEP 8.21 m and R95 14.49 m.

\begin{table}[htbp]
\centering
\caption{Accuracy by environment (meters). From location\_comparison.csv.}
\begin{tabular}{|l|ccc|ccc|}
\hline
\multirow{2}{*}{Environment} & \multicolumn{3}{c|}{Smartphone} & \multicolumn{3}{c|}{Watch} \\
\cline{2-7}
 & CEP & R95 & RMSE & CEP & R95 & RMSE \\
\hline
Ixelles (N=18) & 5.62 & 29.11 & 11.58 & 10.82 & 25.31 & 15.41 \\
LLN (N=24) & 8.21 & 14.49 & 9.88 & 7.02 & 20.21 & 12.41 \\
Waha (N=21) & 3.59 & 9.86 & 6.34 & 5.01 & 10.87 & 7.16 \\
\hline
\end{tabular}
\label{tab:gnss_env}
\end{table}

\begin{figure}[htbp]
\centering
\includegraphics[width=0.8\textwidth]{figure_06_location_heatmap.pdf}
\caption{Error heatmap by location.}
\label{fig:gnss_locations}
\end{figure}

\subsection{Atmospheric conditions}

Cloud cover raises variance. Clear skies (coverage 0) yield smartphone CEP 1.95 m. Overcast (coverage 5) pushes it to 8.21 m. The number of sample might be falsed by outliers but the general trends tend to increase when cloud coverage increase.

\begin{table}[htbp]
\centering
\caption{Accuracy by cloud coverage (0=clear, 5=overcast). From cloud\_coverage\_comparison.csv.}
\begin{tabular}{|c|cc|cc|}
\hline
Coverage & \multicolumn{2}{c|}{Smartphone} & \multicolumn{2}{c|}{Watch} \\
 & CEP & R95 & CEP & R95 \\
\hline
0 & 1.95 & 5.03 & 4.68 & 8.66 \\
1 & 3.76 & 6.39 & 4.45 & 5.26 \\
2 & 7.22 & 29.11 & 10.82 & 24.81 \\
3 & 5.69 & 9.97 & 6.82 & 20.11 \\
4 & 4.31 & 12.16 & 7.61 & 14.19 \\
5 & 8.21 & 12.61 & 8.82 & 20.21 \\
\hline
\end{tabular}
\label{tab:cloud_effects}
\end{table}

\begin{figure}[htbp]
\centering
\includegraphics[width=0.8\textwidth]{figure_04_cloud_coverage_trend.pdf}
\caption{Accuracy by cloud coverage.}
\label{fig:gnss_cloud}
\end{figure}

\section{LTE Positioning (RQ2)}
\label{sec:lte_positioning}

\subsection{Data quality assessment}
\label{subsec:lte_data_quality}

Tower coverage reached 100\% serving cell detection across all 15 experimental runs. Each session confirmed complete database representation for primary connections. Neighbor cell coverage varied widely by environment. Global average stood at 78.6\% across sessions. Ixelles (urban) recorded 79.4\%. LLN (suburban) achieved 92.75\%. Waha (rural) fell to 64.05\%.

\begin{figure}[htbp]
\centering
\includegraphics[width=0.9\textwidth]{figures/coverage_comparison.pdf}
\caption{Tower Database Coverage by Environment.}
\label{fig:tower_coverage_stage1}
\end{figure}

Disparities arise from CellMapper's crowd-sourced nature. Rural areas suffer lower contributor density and fewer recorded neighbor cells~\ref{annex:tower-database}

In Ixelles, session \texttt{ixelle\_5} achieved only 70\% overall coverage with 10 total cells. Suburban LLN sessions averaged 95.3\% overall coverage. 

Rural Waha showed high variability. Sessions \texttt{waha\_1} and \texttt{waha\_2} reached 100\% coverage (4 cells each). Yet \texttt{waha\_3}, \texttt{waha\_4}, and \texttt{waha\_5} recorded 33.33\%, 60\%, and 66.67\% respectively.

Hybrid databases (\texttt{towers.json} (corrected using official Brussels/Wallonia permits) and \texttt{pci.json}) enabled these matches. Strict acceptance rules outperformed raw CellMapper, which previously yielded 1166~m RMSE on 50 ground-truth antennas~\ref{annex:tower-database}


\subsubsection{Geometric Dilution of Precision (GDOP) analysis}

Coverage quality constrains trilateration geometry directly. Ixelles (urban) data shows a bimodal GDOP distribution. About 70\% of solved positions reach GDOP values below 5 due to dense multi-sector tower topology in city centers. The other 30\% record GDOP above 20 in urban canyon segments with collinear towers along single street axes.

Louvain-la-Neuve (suburban) and Waha (rural) datasets display poor geometry consistently. Multiple distinct cells appear per session, yet they originate from only two physical antennas. Near-collinear geometry between these towers generates extreme GDOP values, with means exceeding 2000 in LLN.

Small distance estimation errors ($\epsilon$) amplify into large position errors through this geometry. The relation follows $\text{Position Error} \approx \text{GDOP} \times \epsilon$. Physical tower sparsity forms the main geometric bottleneck in suburban and rural trilateration accuracy.

\subsection{Distance estimation results}
\label{subsec:distance_estimation}

This section evaluates distance estimation models from Stage~\ref{stage:lte-distance-regression}. Cross-validation identifies optimal regression formulas for Ixelles, Louvain-la-Neuve, and Waha environments. Validation against a baseline path-loss model uses real-world field data. Nonlinear regression shows accuracy gains over the baseline.

\subsubsection{Model selection and formula derivation}

Methodology trained regression models to estimate distance $d$ from LTE parameters RSRP, RSRQ, and EARFCN. Target variable was $y = \log_{10}(d_{\text{true}})$.

Table~\ref{tab:lte_distance_best} lists best models per environment from Leave-One-File-Out cross-validation. Selection minimized MAE while limiting model complexity.

Urban Ixelles used quadratic terms for signal power and quality. Suburban LLN and rural Waha included EARFCN interaction terms.

\begin{table*}[htbp]
\centering
\caption{Best distance regression models per environment. MAE and RMSE reported in meters from log-space predictions.}
\hspace*{-1cm}%
\begin{tabular}{|l|l|c|c|}
\hline
\textbf{Environment} & \textbf{Best model features} & \textbf{MAE (m)} & \textbf{RMSE (m)} \\
\hline
Ixelles (Urban) & RSRP, RSRQ, RSRP², RSRQ² & 30.28 & 36.83 \\
\hline
LLN (Suburban) & RSRP, RSRQ, EARFCN, RSRP$\times$RSRQ & 137.50 & 160.60 \\
\hline
Waha (Rural) & RSRP, RSRQ, EARFCN, RSRP$\times$EARFCN & 111.93 & 125.61 \\
\hline
\end{tabular}
\label{tab:lte_distance_best}
\end{table*}

\hspace{}

\textbf{Optimal formulas for Stage 3:}

\begin{itemize}
\item \textbf{Urban (Ixelles):}
\begin{equation}
\log_{10}(d) = 0.92 - 0.018 \cdot \text{RSRP} - 0.051 \cdot \text{RSRQ} - 9.2 \cdot 10^{-5} \cdot \text{RSRP}^2 - 0.0017 \cdot \text{RSRQ}^2
\end{equation}

\item \textbf{Suburban (LLN):}
\begin{equation}
\log_{10}(d) = 2.67 + 0.0013 \cdot \text{RSRP} + 0.0079 \cdot \text{RSRQ} - 0.0003 \cdot \text{EARFCN}_k + 7.1 \cdot 10^{-5} \cdot (\text{RSRP} \cdot \text{RSRQ})
\end{equation}

\item \textbf{Rural (Waha):}
\begin{equation}
\log_{10}(d) = 0.94 - 0.0004 \cdot \text{RSRP} - 0.013 \cdot \text{RSRQ} + 1.23 \cdot \text{EARFCN}_k - 0.0005 \cdot (\text{RSRP} \cdot \text{EARFCN}_k)
\end{equation}
\end{itemize}


\subsubsection{Validation against path loss}

To verify the robustness of these formulas, we implemented them in the full geolocation pipeline (Stage 3) and compared their distance estimates against a standard calibrated Path Loss model. The validation was performed across all collected traces, comparing the estimated distance to the ground truth Haversine distance between the serving cell tower and the device's GNSS position.

The formula-based approach demonstrated a significant improvement in estimation accuracy over the traditional path loss model. As shown in Figure \ref{fig:method_comparison}, the global Mean Absolute Error (MAE) dropped from approximately 693 meters with the Path Loss model to just 91 meters with the Formula method.

\begin{figure}[htbp]
    \centering
    \includegraphics[width=0.8\textwidth]{distance_mae_comparison.pdf}
    \caption{Global comparison of distance estimation error between Path Loss and Formula methods.}
    \label{fig:method_comparison}
\end{figure}

The Root Mean Square Error (RMSE) showed a similar trend (752 m vs 91 m). By removing the extreme outliers present in previous iterations, the baseline path loss model now reflects a more realistic performance (MAE $\approx$ 690-760 m), yet it remains nearly an order of magnitude less accurate than the formula-based approach.

The performance advantage of the Formula method is consistent across all three urbanization levels:

\begin{itemize}
    \item \textbf{Ixelles (Urban):} The path loss model yielded an MAE of roughly 691 m. The density of urban towers ensures strong signals but also introduces complex reflections. The Formula method successfully modeled these complexities, reducing the MAE to 93 m.
    \item \textbf{Louvain-la-Neuve (Suburban):} This environment remained the most challenging for the path loss model, with an MAE of 759 m. The complex mix of pedestrian zones and vegetation likely caused significant signal variance. The Formula method remained highly robust, achieving its best performance here with an MAE of 82 m.
    \item \textbf{Waha (Rural):} The Formula method achieved an MAE of 137 m compared to the path loss baseline of 390 m. While the absolute improvement is smaller in meters compared to urban zones, the relative accuracy gain is still substantial ($\approx$ 65\% reduction in error).
\end{itemize}

A deeper analysis of per-cell errors demonstrates the stability of the regression approach. Figure \ref{fig:error_boxplot} presents the distribution of MAE for individual cell towers.

\begin{figure}[htbp]
    \centering
    \includegraphics[width=0.8\textwidth]{distance_error_distribution.pdf}
    \caption{Box plot of per-cell distance estimation errors. The Path Loss method exhibits high variability (interquartile range spanning hundreds of meters). The Formula method produces a tight, consistent error distribution with errors concentrated below 100 meters.}
    \label{fig:error_boxplot}
\end{figure}

The path loss data is characterized by high variability, with an interquartile range spanning several hundred meters. In contrast, the Formula method constrains the error distribution significantly. The median error is lower, and the spread (variance) is much tighter, ensuring that no single cell introduces a massive bias into the positioning system.

This consistency is critical for the subsequent trilateration stage. Even without extreme multi-kilometer outliers, the path loss model's variance (RMSE $\approx$ 750 m) is large enough to degrade positioning accuracy. The Formula method's ability to keep distance errors consistently low ($<$ 140 m on average) provides the high-quality inputs necessary for the precise trilateration discussed in the next section.

\subsection{Trilateration and positioning results}
\label{subsec:trilateration_results}

Geolocation pipeline evaluated final positioning performance across three urbanization environments. Stage 2 distance estimates compared against Log-Distance Path Loss baseline for latitude/longitude accuracy.

\subsubsection{Data completeness and sample statistics}
\label{ssub:data_stats}
Trilateration solver produced 2708 valid position estimates from 14 sessions. Ixelles generated 524 samples across 4 sessions. Louvain-la-Neuve yielded 1910 from 5 sessions. Waha provided 274 from 5 sessions. Sessions ixelle\_3, lln\_4, waha\_1 retained 2 of 8 contexts due to signal loss. Session waha\_5 captured 6 contexts after filtering. Eight contexts combined path\_loss and formula-based methods with city, town, village, default classifications.

\subsubsection{Overall performance comparison}

Table~\ref{tab:trilateration_overall} summarizes aggregate metrics across environments.

\begin{table}[htbp]
\centering
\caption{Aggregate trilateration performance comparison (All Environments, N=2,708).}
\begin{tabular}{lcccc}
\toprule
\textbf{Method} & \textbf{RMSE (m)} & \textbf{Mean Error (m)} & \textbf{Std Dev (m)} & \textbf{R95 (m)} \\
\midrule
Path Loss (Baseline) & 1181.0 & 1145.9 & 253.9 & 1360.5 \\
Regression Formula & 537.1 & 527.1 & 85.6 & 646.7 \\
\bottomrule
\end{tabular}
\label{tab:trilateration_overall}
\end{table}

Formula method reduced RMSE from 1181~m to 537~m (54.5\% improvement). Standard deviation dropped from 254~m to 86~m (66\% reduction).[21]

\subsubsection{Performance by environmental context}

Figure~\ref{fig:trilateration_rmse_context} compares RMSE across locations. Trilateration accuracy depends on distance estimation, network geometry, and tower availability. These factors create unique error profiles in each environment.

\begin{figure}[htbp]
\centering
\includegraphics[width=0.8\textwidth]{figures/trilateration_accuracy.pdf}
\caption{RMSE comparison by environment, showing the impact of network geometry and model selection.}
\label{fig:trilateration_rmse_context}
\end{figure}

\paragraph{Urban environment (Ixelles)}
In Ixelles, both positioning methods achieve 166.0~m RMSE. High tower density creates short baselines between 200 and 800~m. This network density enables mutual error cancellation. Measurements from multiple angles counteract systematic multipath errors. Redundant data from approximately eight cells per epoch ensures well-conditioned geometry. Specifically, GDOP values remain at or below 8.0 in 70\% of positions, as detailed in Section ~\ref{ssub:data_stats}. Urban geometry defines the error floor here. Strict three-cell filtering excludes canyon segments with GDOP above 20.0.

\paragraph{Suburban environment (Louvain-la-Neuve)}
Suburban accuracy relies on distance model quality due to sparse tower distribution. The Path Loss model averages 1181~m RMSE. In contrast, the Formula method reaches 537~m RMSE. This 54.5\% improvement results from empirical regression capturing local propagation in mixed zones. Collinear tower placement in LLN causes high GDOP, which averages 1847. High GDOP amplifies small distance errors into kilometer-scale position deviations. The Formula method reduces distance Mean Absolute Error (MAE) from 759~m to 137.5~m. Accurate distance estimation significantly lowers the resulting geometric penalty.

\paragraph{Rural environment (Waha)}
Rural results converge at 691~m RMSE for both methods. Geographic and data completeness constraints dominate performance. Sparse tower distribution and database gaps limit neighbor cell availability.
\begin{itemize}
\item Coverage in sessions Waha\_3 to Waha\_5 ranges from 33\% to 66\% according to Section 5.2.1.
\item The solver often relies on only 1 or 2 reference points due to these database gaps.
\item Rural towers cluster along major roads, creating collinear geometry and high GDOP values.
\end{itemize}
Intersection error from two circles dominates distance uncertainty. Convergence at 691~m indicates that data scarcity determines the error limit in this environment. Geography dictates performance over methodology in these regions.

\subsubsection{Cell availability and accuracy coupling}

Suburban formula contexts varied by residential classification: village (290 samples, 506.2~m RMSE), town (284 samples, 549.1~m RMSE).[21] Urban village context reached 166.0~m RMSE across 135 samples for both methods. Rural convergence occurred at 691~m RMSE despite Stage 2 formula improvements.

Trilateration accuracy limited by minimum of distance model quality and network geometry.

\subsubsection{Forensic reliability assessment}

R95 confidence interval reduced from 1360~m (Path Loss) to 646~m (Formula). Empirical modeling prevented catastrophic suburban failures common in forensic scenarios.

\chapter{Conclusion}
\label{ch:conclusion}

\section{Key findings}
\label{sec:conclusion_findings}

This thesis examined smartphone geolocation reliability across three Belgian environments. The analysis addressed two research questions. RQ1 assessed GNSS accuracy against ground truth references. RQ2 evaluated LTE network positioning without GPS data.

\subsection*{RQ1: GNSS accuracy across environmental contexts}

Environment determines GNSS location trace quality for forensic use. The OnePlus 6T produced a global CEP of 5.61\,m across all measurements. Results varied significantly by location type.

Rural Waha positioning reached CEP 3.59\,m and RMSE 6.34\,m. These values support precise timeline reconstruction in investigations.

Urban Ixelles achieved median CEP 5.62\,m, matching rural medians. Outlier errors reached R95 29.11\,m due to canyon multipath effects.

Suburban Louvain-la-Neuve showed CEP 8.21\,m and R95 14.49\,m. Mixed of building density and narrow street created intermediate challenges.

Smartwatch comparisons confirmed reliability. 68.0\% of synchronized measurements stayed within 10\,m. Median separation measured 7.7\,m. Dual consumer devices thus corroborate geolocation evidence effectively.

\subsection*{RQ2: LTE network positioning as GPS-fallback}

LTE signal strength enables location estimation without satellite data. Investigators apply this method when GPS fails or devices disable positioning. Analysis created environment-specific RSRP distance models.

Models to evaluate distance to the cell achieved MAE 30.28\,m in urban areas, 137.50\,m in suburban zones, and 111.93\,m in rural locations.

Trilateration using these models outperformed path loss baselines. Suburban RMSE dropped 54.5\% from 1181\,m to 537\,m. 95\% confidence radius improved from 1360\,m to 647\,m.

Urban trilateration matched baseline RMSE at 166\,m. Tower geometry limited rural performance to RMSE 691\,m in both methods.

LTE positioning reveals movement patterns across coverage areas. Evidence excludes impossible travel timelines through tower handoffs. General area presence becomes confirmable. Precise crime scene claims require additional corroboration.


\subsection*{Forensic evidence interpretation framework}

Casey's three-phase framework guides location trace analysis through production, persistence, and investigation stages. This research classifies evidence by source reliability:

\textbf{Primary evidence:} GNSS coordinates from smartphone logs or forensic extraction serve as primary traces. Quantified CEP/R95 values establish accuracy margins for timelines and proximity claims. Environment-stratified metrics communicate confidence directly to investigators and courts.

\textbf{Secondary evidence:} LTE positioning derives from routine signal measurements. These traces support movement patterns and exclusion claims. Precise location assertions demand corroboration from other sources.

\textbf{Environmental context:} Environment classification determines accuracy weight. Rural GNSS positioning carries greater evidentiary value than urban GNSS. Urban R95 outliers reach 29\,m. Multipath uncertainty accompanies median accuracy statements in testimony.

\subsection*{Research question Rresolution}

\paragraph{RQ1 Resolution}
GNSS accuracy varies with environment. Rural areas deliver CEP 3.59\,m. Urban medians produce CEP 5.62\,m despite outlier challenges. Suburban conditions yield CEP 8.21\,m. Belgian courts gain empirical baselines for positioning evidence admissibility through these quantified metrics.

\paragraph{RQ2 Resolution}
LTE trilateration reconstructs position absent GPS signals. Results show environmental dependence. Suburban analysis achieves RMSE 537\,m alongside 95\% radius 647\,m. Urban tower geometry maintains RMSE 166\,m. Rural cell sparsity limits positioning to broad area confirmation. Investigators combine methods for robust forensic conclusions.


\section{Contributions}
\label{sec:conclusion_contributions}

This work delivers three advances in smartphone forensic geolocation. Measurements establish environmental-stratified GNSS accuracy baselines. Regression-based LTE trilateration quantifies network positioning limits. Open-source code enables investigator replication.

\subsection*{Contribution 1: Belgian-stratified GNSS accuracy baseline}

No prior study measures smartphone GNSS accuracy systematically across Belgian urbanization levels. Isolated measurements exist. \cite{merry2019smartphone} report 9.9\,m RMSE urban. \cite{tomastik2017horizontal} report 7.48\,m forest leaf-on and 2.11\,m open sky. \cite{oluwadare2024comparative} report 3.37\,m cadastral. None address suburban accuracy. Forensic cases span all environment types.

This thesis provides first Belgian measurements with five-session validation per location. Rural CEP measures 3.59\,m. Urban CEP measures 5.62\,m. Suburban CEP measures 8.21\,m. Ground truth combines smartwatch data, EXIF coordinates, and physical waypoints. OnePlus 6T running Android 15 captures current GNSS capabilities.

\subsection*{Contribution 2: First LTE regression-based trilateration for forensics}

Prior LTE research focuses on signal prediction. \cite{alimpertis2019city} predict RSRP distributions (4.64\,dB RMSE). \cite{guggenheimer2024global} applies RF fingerprinting to cell-ID matching. \cite{prihodko2018machine} predicts RSRP time-series for handover optimization. No studies validate trilateration positioning accuracy. No studies quantify geographic error margins.

This thesis develops empirical RSRP-to-distance regression models. Validation uses ground-truth coordinates through trilateration. Three environment-specific models emerge:

\begin{itemize}[noitemsep]
\item Urban: MAE 30.28\,m
\item Suburban: MAE 137.50\,m
\item Rural: MAE 111.93\,m
\end{itemize}

Trilateration reduces RMSE 54.5\% versus path loss baseline (1181\,m $\to$ 537\,m). Confidence radius improves from 1360\,m to 647\,m at 95\%. Variance reduction reaches 66\%. Suburban baseline fails at 1181\,m error; regression achieves 537\,m. Haversine distances validate formulas. Field tests cover three environments.

\subsection*{Contribution 3: Open-Source framework for investigators}

Modular Python framework enables geolocation validation. Three independent pipelines execute GNSS comparison, distance regression, and trilateration. CLI interface serves non-specialists. Tower database construction follows Annex ~\ref{annex:tower-database} template.

Dataset appears on GitHub under MIT license. Smartphone measurements, smartwatch references, and cell tower locations accompany code. Scripts generate publication-quality plots and statistical tables. Documentation covers Sections 4.3 and Annex A.

Local police apply framework to local networks. Investigators train models on regional data and establish testimony standards. Code produces CEP circles and error distributions for court use. Framework adapts to new locations without modification.


\section{Limitations}
\label{sec:conclusion_limitations}

This thesis establishes GNSS and LTE baselines within defined experimental scope. Internal validity holds firm. External validity faces multiple constraints. Practitioners apply results within documented boundaries.

\subsection*{Experimental design constraints}

Single device testing limits generalizability. Only OnePlus 6T with LineageOS 22.2 (Android 15) underwent evaluation. Antenna design differences across manufacturers affect GNSS accuracy (Chapter 1.1.3.2).

\begin{itemize}[noitemsep]
\item Results exclude iPhone, Samsung Galaxy, Google Pixel devices
\item Dual-frequency flagships (iPhone 15 Pro, Pixel 8 Pro) show different performance  
\item Each device family requires independent forensic validation
\end{itemize}

Data collection occurred during 2025 academic year. GNSS constellation, Android 15 algorithms, and LTE infrastructure represent specific temporal snapshot. Continuous changes affect future applicability.

\begin{itemize}[noitemsep]
\item Constellation expansion alters GDOP characteristics
\item Android 16 may modify positioning algorithms
\item Tower deployments shift monthly across Belgium
\item Seasonal foliage changes impact multipath unmeasured here
\end{itemize}

Field sessions lasted 15--30 minutes during daytime walking. Overnight positioning, stationary behavior, and diurnal satellite variation remain uncharacterized.

\begin{itemize}[noitemsep]
\item Daytime walking metrics exclude nighttime accuracy
\item 24-hour alibis need overnight validation
\item Stationary devices (vehicles, indoors) produce different error profiles
\end{itemize}

Garmin Venu Sq 2 smartwatch provides consumer-grade reference (5-10\,m accuracy). Professional surveying equipment (±1\,m) absent from validation chain (Chapter 1.2.1).

\begin{itemize}[noitemsep]
\item CEP values below 10\,m reflect reference limitations
\item Multipath affects both devices simultaneously
\item Survey-grade baseline needed for absolute accuracy claims
\end{itemize}

\subsection*{Validation \& ground truth limitations}

EXIF waypoints derive from same smartphone GNSS receiver. Validation captures relative drift, not absolute systematic bias. OS algorithm errors and multipath affect both reference and test measurements equally.

\begin{itemize}[noitemsep]
\item Systematic bias detection impossible through circular validation
\item Relative precision measurable; absolute accuracy unproven
\item Professional RTK GNSS (±0.5\,m) establishes true ground truth
\end{itemize}

\subsection*{LTE positioning methodology limitations}

Tower database covers 79.4--92.7\% of serving cells. 12-15\% lack coordinate data.

\begin{itemize}[noitemsep]
\item Positioning fails below 80\% tower coverage
\item Rural RMSE 691\,m reflects partial database failure
\item Neighbor visibility (60--90\%) constrains trilateration geometry
\end{itemize}

Path loss exponent fixed per frequency band. Model ignores terrain, vegetation, building materials.

\begin{itemize}[noitemsep]
\item MAE varies 2--3× across individual cells
\item Seasonal foliage alters propagation unquantified
\item Adaptive calibration improves distance estimation
\end{itemize}

RSRQ infers bearing confidence without validation. Multipath produces high RSRQ despite poor directionality.

\begin{itemize}[noitemsep]
\item Bearing errors (10-30\textdegree) uncalibrated
\item Angle-of-arrival hardware needed for precision
\end{itemize}

Regression formulas overfit to three test locations. Retraining required for new regions.

Timing Advance parameter excluded from analysis. TA measures direct round-trip delay, bypassing path loss uncertainty.

Three locations insufficient for national baseline. Walloon results exclude Flemish, coastal, and Ardennes environments.


\section{Future directions}
\label{sec:conclusion_future}

Future research should address these limitations through:
\begin{itemize}[noitemsep]
    \item \textbf{Cross-OS analysis:} Android vs iOS raw data accessibility and forensic reconstruction differences
    \item \textbf{Vehicular testing:} Higher-speed measurements capturing frequent cell handovers for improved trilateration
    \item \textbf{Multi-operator:} Proximus/Orange/Telenet comparison for PCI overlap and network effects
    \item \textbf{Deep learning optimization:} Neural networks to auto-tune triangulation parameters and mitigate NLOS
    \item \textbf{Hybrid positioning:} GPS + WiFi/Bluetooth fusion accuracy quantification
    \item \textbf{Tower enhancement:} Antenna patterns, real-time CellMapper integration, 3D occlusion modeling
    \item \textbf{Provider azimuth data integration:} Access operator tower antenna bearings and sector boundaries. Requires operator data standardization across Belgian providers and legal framework for evidence access.
    
\end{itemize}

\newpage

\section{Final remarks}
\label{sec:conclusion_remarks}
Completing this thesis felt much like hunting GNSS satellites in an urban canyon: technically challenging, frequently frustrating, and deeply rewarding when you finally achieve lock.

Device selection proved the primary challenge. The Samsung Galaxy A51's encryption, OnePlus Nord2's deleted firmware, and Huawei P8's outdated Android taught a hard lesson: methodology matters little without the right hardware. Walking predetermined routes across Brussels, Louvain-la-Neuve, and Waha in -5°C rain, I probably looked like a suspicious character to my neighbors. Yet these field sessions proved invaluable, anchoring thousands of ground truth measurements.

The findings reveal smartphone GNSS positioning with median errors of 4.9 m in open-sky, 8.7 m in suburban areas, and 15.2 m in urban canyons. LTE-based positioning achieved 166m accuracy in cities but converged at 691m in rural zones, revealing that geography enforces hard limits no algorithm can violate. These results demonstrate consumer-grade smartphones provide forensically relevant accuracy across diverse Belgian environments.

This thesis contributes a necessary foundation for future research. Multiple device platforms, seasonal variations, and indoor scenarios remain unexplored. The hope is that this work eventually leads to comprehensive, standardized guidelines for smartphone positioning evidence in Belgian courts.

To future researchers: prioritize device selection and validate relentlessly against multiple ground truth sources. And when multipath errors reach 29 meters in urban canyons, remember that forensics tolerates imperfection beautifully as long as uncertainty is explicit.

% --------------------------------------------------
%                BIBLIOGRAPHY
% --------------------------------------------------
\clearpage
\printbibliography[heading=bibintoc]

% --------------------------------------------------
%                ANNEXES
% --------------------------------------------------
\appendix
\chapter{Annexes}

\section{Device reset and preparation procedure}
\label{annex:device-prep}

This annex details the exact protocol for resetting the OnePlus 6T (LineageOS 22.2 / Android 15) before each data collection session.

\subsection{Initial setup}
\begin{enumerate}
\item Perform factory reset.
\item Skip Mobile Network and Wi-Fi during OOBE.
\item Set date/time manually if needed.
\item Enable Google Services.
\item Set PIN to 1234.
\item Skip fingerprint and data restore.
\item Complete LineageOS setup.
\item \textit{If Google Services enablement fails:} Grant permissions via:
\begin{verbatim}
adb shell pm grant com.google.android.gms android.permission.ACCESS_COARSE_LOCATION
adb shell pm grant com.google.android.gms android.permission.ACCESS_FINE_LOCATION
\end{verbatim}
\end{enumerate}

\subsection{Connectivity}
\begin{enumerate}
\item Enable Geolocation for camera app.
\item Connect to Wi-Fi.
\item Sign into project Google Account (no backup restore).
\item Restart device if prompted.
\item Open Google Maps. Authorize "While using app" location.
\item Verify location lock in Maps.
\end{enumerate}

\subsection{Developer options}
\begin{enumerate}
\item Enable Developer Options.
\item Disable Wi-Fi Scanning and Bluetooth Scanning.
\item Enable USB Debugging and Rooted Debugging.
\end{enumerate}

\subsection{Rooting}
\begin{enumerate}
\item Connect to PC. Allow file transfer.
\item Copy magiskv29.0.apk and gnettrackpro.375.apk to storage.
\item Disconnect USB.
\item Install Magisk APK. Complete prompts. Reboot.
\item Verify root: adb shell, then su. Grant Magisk popup.
\end{enumerate}

\subsection{G-NetTrack Pro configuration}
\begin{enumerate}
\item Install gnettrackpro.375.apk (bypass Play Protect).
\item Launch app. Grant Always Allow location.
\item Settings > Log Parameters:
\begin{itemize}
\item Timer Interval: 1s
\item Include IMSI, IMEI, MSISDN: Enabled
\item Verbose Log: Enabled
\item Write Cell Info in Log: Enabled
\item Ask for Log Name: Enabled
\end{itemize}
\end{enumerate}


\section{Tower database construction}
\label{annex:tower-database}

Tower coordinates determine LTE trilateration accuracy. I built \texttt{towers.json} and \texttt{pci.json} through validation.

\subsection{Dataset structure}
\texttt{towers.json} uses Cell ID as key. Each entry holds \texttt{enb\_id}, corrected \texttt{latitude} and \texttt{longitude}, \texttt{pci}, \texttt{band}, \texttt{earfcn}, \texttt{azimuth}, and field signal data.

\texttt{pci.json} uses PCI as key. Each entry holds \texttt{pci}, \texttt{latitude}, \texttt{longitude}, \texttt{band}, \texttt{operator}, and \texttt{frequency\_mhz}.

\subsection{CellMapper problems}
CellMapper supplied initial coordinates. Positioning errors remained high.

\subsection{Quality check}
Charleroi police supplied operator reports with addresses. I geolocated 50 antennas using Google Street View for ground truth.

I have written a scripts that calculated Haversine errors. \texttt{cellmapper\_qa.py} reported RMSE 1166 m, median 471 m, max 3153 m. \texttt{extract\_outliers.py} flagged errors over 1000 m.

41 of 50 cells had estimates. Mean error reached 834 m.

\subsection{Official sources}
Brussels permits came from geodata.environnement.brussels. Wallonia permits came from geoportail.wallonie.be.

\subsection{Mapping steps}
I matched official antennas to CellMapper eNBs. Position errors stayed below RMSE. Frequency bands aligned. Azimuths matched. CellMapper first-seen dates followed official installs. Addresses confirmed municipality and street.

\subsection{Population steps}
I assigned official \texttt{latitude} and \texttt{longitude} to matched eNBs in \texttt{towers.json}. CellMapper populated \texttt{enb\_id}, \texttt{pci}, \texttt{azimuth}, \texttt{band}, \texttt{earfcn}. G-NetTrack logs added signal data. I linked \texttt{pci.json} entries to corrected coordinates.

\subsection{Acceptance rules}
Official coordinates must exist. Matching passes all four criteria. I review errors over 1000 m.

\subsection{Remaining errors}
Official permits outperform raw CellMapper data. Antenna panel offsets create residual errors. Permit coordinates carry imprecision. Hybrid method improves positioning over CellMapper alone.





% Back cover page
\backcoverpage

\end{document}
