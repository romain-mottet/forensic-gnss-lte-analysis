\chapter{Methodology}
\label{chap:methodology}

This chapter evaluates the framework from Chapter~3 against the research questions RQ1 and RQ2 defined in Section~\ref{sec:intro_research_questions}. 
RQ1 measures smartphone GNSS accuracy by comparing device coordinates with Garmin Venu Sq~2 references and EXIF-derived waypoints, while RQ2 validates LTE signal-based distance models used for trilateration in the absence of GPS. 

The analysis relies on 15 data-collection sessions conducted across 3 Belgian environments. 
Urban measurements took place in Ixelles (Brussels Capital Region), suburban routes followed predefined paths in Louvain-la-Neuve (Brabant Wallon), and rural tracks were recorded around Waha (Luxembourg Province). 
Each $\approx$ 800~m walk included five fixed waypoints: start, photo1, photo2, photo3, and end, ensuring repeated observations at the same physical locations. 

Sessions occurred on different dates and at varied times of day in order to capture changes in satellite geometry, weather conditions, and network behaviour.

\section{Experimental setup}

The test configuration follows the final device selection from Section~\ref{sec:exp_device_journey}. Forensic reproducibility guides this setup.

\begin{itemize}
\item \textbf{Smartphone:} OnePlus 6T runs LineageOS 22.2 (Android 15). Magisk v29 provides root access. G-NetTrack Pro v0.375 logs networks at 6~s intervals. Logs include IMSI/IMEI/MSISDN and full cell data in verbose mode. Camera photos embed EXIF GPS metadata. These serve as primary forensic ground truth.
\item \textbf{Smartwatch:} Garmin Venu Sq 2 delivers independent GNSS reference. GPX track export enables cross-validation. Smartphone Android location APIs stay excluded.

\item \textbf{Forensic workstation:} ADB root extraction produces tar.gz filesystem archives. Autopsy 4.22.1 processes these with aLEAPP. Output creates ground\_truth\_waypoints.csv. This file holds EXIF coordinates, timestamps, and session metadata tied to physical locations.
\end{itemize}

A strict protocol prepares devices before each series. Factory reset clears persistent state. Disabled WiFi/Bluetooth scanning blocks assisted positioning artifacts. Manual time sync aligns timestamps across devices.

See Annex~\ref{annex:device-prep} for full preparation details.

\section{Field data collection protocol}
Each session follows a standardized procedure ensuring comparability across environmental conditions and repetitions:

\begin{enumerate}
    \item \textbf{Initialization:} Confirm GPS lock accuracy $\leq$10~m, launch G-NetTrack Pro logging, start Garmin activity recording, and document cloud coverage on 0--5 scale (0=clear sky, 5=overcast) as primary environmental covariate.
    \item \textbf{Route execution:} Traverse waypoints in fixed sequence taking timestamped photographs (EXIF ground truth); between photo2 and photo3, activate Google Maps navigation to a nearby shop simulating realistic foreground application usage that influences background location behavior.
    \item \textbf{Repetition:} Conduct five sessions per environment on different days/times capturing diurnal satellite geometry variations and network load differences.
\end{enumerate}

This protocol generates synchronized multimodal datasets: raw LTE/GPS logs, Garmin GPX tracks, forensic EXIF waypoints, and environmental metadata enabling comprehensive geolocation quality assessment.

\section{Analysis pipeline}
Figure~\ref{fig:diagram} shows the three-stage workflow. Raw field data transforms into forensic positioning metrics. Custom scripts from Chapter 3 process EXIF waypoints, GPS comparisons, LOFO regression, and LTE trilateration.

\begin{figure}[htbp]
    \centering
    \hspace*{-2cm}
    \includegraphics[width=1.2\textwidth]{figures/diagram.pdf}
     \caption{Three-stage pipeline: field collection → forensic extraction → Different python script analysis.}
    \label{fig:diagram}
\end{figure}

\subsection{Stage 1: GPS comparison pipeline (12 Steps)}

The GPS analysis pipeline executes twelve sequential steps to validate and compare smartphone and Garmin watch GPS tracks against ground-truth waypoints. These steps convert raw data, align timestamps optimally, flag quality issues, compute accuracy metrics, and generate visualizations.

Parameters such as time windows (0.5--8.0~s), offsets ($\leq$3600~s), gaps ($\geq$12~s unstable, $\geq$60~s major), waypoint matching (25~s), and phone-watch agreement (10~m) are tuned for this experiment but can be adjusted based on context and desired outcomes. The pipeline code is modular, with all key parameters centralized in a dedicated configuration file (src/config/parameters.py) that allows easy modifications without altering the core implementation.

\begin{enumerate}
    \item \textbf{GPX to CSV conversion}: Recursively locate all Garmin watch .gpx files and convert them to CSV format (gps\_groundtruth.csv), preserving timestamps, coordinates, and accuracy fields like HDOP where available.
   
    \item \textbf{Optimal time window detection}: For each smartphone-watch pair, test alignment windows from 0.5 to 8.0 seconds in 0.5-second increments to identify the window maximizing match density.
   
    \item \textbf{Smartphone-watch merging}: Merge datasets using the optimal window and correct clock offsets up to 3600 seconds (1 hour) to handle DST changes; reject pairs exceeding this limit. Output per-session comparison files.
   
    \item \textbf{Quality gap analysis}: Scan merged files for duplicate timestamps and gaps, flagging unstable gaps $\geq$12 seconds and major gaps $\geq$60 seconds. Generate quality reports.
   
    \item \textbf{Data flagging}: Add quality flags to comparison data, including ``iscacheduplicate'' for duplicates and ``nearmajorgap30s'' for records within 30 seconds of major gaps.
   
    \item \textbf{Unified dataset creation}: Concatenate flagged files across sessions, enrich with metadata like cloud coverage from waypoints, and produce unified\_gps\_dataset.csv.
   
    \item \textbf{Waypoint distance matching}: Match GPS fixes to ground-truth waypoints within 25-second temporal windows, computing Haversine distances between coordinates. Generate per-session accuracy reports.
   
    \item \textbf{Global device accuracy}: Aggregate waypoint distances to compute per-device metrics: CEP (50th percentile), R95 (95th percentile), and RMSE.
   
    \item \textbf{Accuracy by location}: Stratify accuracy metrics (CEP, R95, RMSE) by location for environmental context.
   
    \item \textbf{Accuracy by cloud coverage}: Group metrics by cloud coverage categories (0=clear to 5=overcast) to assess environmental impact.
   
    \item \textbf{Phone-Watch agreement}: At aligned timestamps, flag instances where both devices report positions within 10 meters of each other.
   
    \item \textbf{Visualization generation}: Produce six PDF figures, including accuracy heatmaps, scatter plots, boxplots, and trends by location, device, and cloud coverage.
\end{enumerate}

\subsection{Stage 2: LTE distance regression}
\label{stage:lte-distance-regression}

This pipeline estimates LTE cell tower distances from signal measurements (RSRP, RSRQ, EARFCN) using parsed logs from stage 3. It associates measurements with ground-truth tower locations, engineers features, performs Leave-One-File-Out (LOFO) cross-validation, selects optimal models, and exports formulas in json formula directly usable in stage 3.

\begin{enumerate}
    \item \textbf{Input data ingestion}: Load parsed LTE measurement CSV files (parsed\_*.csv) containing timestamps, GPS coordinates, RSRP, RSRQ, PCI, and EARFCN for serving/neighbor cells.
   
    \item \textbf{Tower database joining}: Match measurements to tower locations using primary key (PCI + EARFCN) from towers.json; fallback to PCI-only matching via pci.json if enabled. Flag and exclude unmatched entries.
   
    \item \textbf{Ground-truth distance computation}: Calculate Haversine distances \(d_{\text{true}}\) (in meters) between receiver GPS and matched tower coordinates.
   
    \item \textbf{Log-space target framing}: Transform distances to \(y = 10 \log_{10}(d_{\text{true}})\) to stabilize variance, enforce positivity, and align with log-distance path loss models.
   
    \item \textbf{Feature engineering}: Prepare linear features (RSRP, RSRQ, EARFCN$_k$/1000) or nonlinear expansions (squares: rsrpsq/rsrqsq; interaction: rsrpxearfcnk). Limit to max 18 neighbors (limit from G-NetTrack Pro), per timestamp, filtering non-LTE.
   
    \item \textbf{LOFO Cross-Validation}: Perform Leave-One-File-Out validation rotating across measurement files per environment/context to test generalization.
   
    \item \textbf{Model evaluation and selection}: Fit linear regression models per fold; select the best minimizing MAE, RMSE, and MAPE across baseline and nonlinear variants. Log detailed per-fold metrics.
   
    \item \textbf{Formula export}: Save optimal model as formula\_*.json with coefficients (e.g., intercept 0.9427, coefrsrp -0.00038), features, and notes for downstream use.
\end{enumerate}


\subsection{Stage 3: LTE network positioning (9 Phases)}

The LTE positioning pipeline runs nine phases. Four environmental contexts guide it: \textbf{default}, \textbf{town}, \textbf{city}, \textbf{village}. See Section~\ref{subsec:env_contexts}. Stage 2 formula variants feed Phase 2.2 for distance estimates. These enable model comparisons. Outputs match Haversine distances and GPS ground truth. Path-loss random errors beat formula precision in trilateration. Bias explains this.

\subsubsection{Environment contexts: assumptions and parameter calibration}
\label{subsec:env_contexts}

Propagation environment shapes LTE accuracy. Signal geometry matters too. Four contexts handle this: \textbf{default}, \textbf{city}, \textbf{town}, \textbf{village}. Each sets parameters for signal traits and cell density. Contexts set Phase 2.2 distance uncertainty. They also control Phase 3.1--3.2 trilateration limits.

\paragraph{Path Loss exponent and distance estimation}

All contexts use this path-loss model:
\begin{equation}
d = 10^{\frac{\text{RSRP}{\text{ref}} - \text{RSRP}{\text{meas}}}{10n} + h(\text{RSRQ})}
\end{equation}

$n$ is the path-loss exponent. It spans 2.7-2.82 for LTE bands 700--2600 MHz. ITU-R M.1225 and 3GPP TS 36.942 provide data~\cite{ITU-R:2008, 3GPP:2023}. Exponents stay fixed across contexts. Contexts adjust distance uncertainty bounds instead. RSRP quality sets these bounds. Band physics reuse holds. Environmental signal stability varies.

\paragraph{Signal quality and uncertainty classification}

RSRP falls into six levels. Thresholds apply everywhere:
\begin{itemize}
\item \textbf{EXCELLENT} ($\text{RSRP} > -85 \text{ dBm}$): Near-tower. Fading variance $\pm$2--5 dB.
\item \textbf{HIGH} ($-85$ to $-100$ dBm): Moderate multipath. Fading variance $\pm$5--10 dB.
\item \textbf{GOOD} ($-100$ to $-110$ dBm): Non-line-of-sight. Fading variance $\pm$8--12 dB.
\item \textbf{MEDIUM} ($-110$ to $-120$ dBm): Multipath and shadowing. Fading variance $\pm$15--20 dB.
\item \textbf{WEAK} ($-120$ to $-130$ dBm): Coverage edge. Variance unpredictable.
\item \textbf{VERY_WEAK} ($< -130$ dBm): Extreme edge. Estimates fail.
\end{itemize}

Max uncertainty in meters shifts by context. Cities use 80-250 m. Dense towers help. Villages use 100-400 m. Sparse coverage demands this. Towns take middle values.

\paragraph{City context}

High tower density overlaps coverage. Strong signals select easily. Max cells $= 8$. $\text{GDOP}_{\text{max}} = 8.0$.
\begin{itemize}
\item $> 6$ cells visible. Pick 4-8 strongest.
\item Urban canyons raise $n \approx 2.8$. Matches 3GPP UMi.
\item Filter RSRP at $-110$ dBm.
\item GDOP $\leq 8.0$. Convergence 0.5 m.
\item Bounds cut 20 m from default. HIGH: 100 m.
\end{itemize}

\paragraph{Town context}

Moderate density mixes line-of-sight and non-line-of-sight. Max cells $= 6$. $\text{GDOP}_{\text{max}} = 9.0$.
\begin{itemize}
\item 3-6 cells visible.
\item Multipath moderate.
\item Filter RSRP at $-115$ dBm.
\item GDOP $\leq 9.0$. Convergence 0.75 m.
\item Bounds cut 10--20 m. HIGH: 110 m.
\end{itemize}

\paragraph{Village context}

Sparse coverage means large cells. Max cells $= 5$. $\text{GDOP}_{\text{max}} = 10.0$.
\begin{itemize}
\item 2--5 cells visible. Min cells $=2$.
\item Terrain raises $n \approx 3.5$-$4.0$. Fits 3GPP RMa.
\item Filter RSRP at $-120$ dBm.
\item GDOP $\leq 10.0$. Convergence 1.0 m.
\item Bounds match default. HIGH: 120 m.
\end{itemize}

\paragraph{Default context}

Baseline matches village. Max cells $= 5$. $\text{GDOP}_{\text{max}} = 10.0$. Use without context.

\paragraph{Bearing and trilateration weighting}

Solver weights by $w_i = 1/\sigma_i^2$. Phase 2.2 sets $\sigma_i$. This favors strong cells. GDOP filters bad geometry post-solution.

\paragraph{Parameters setup}

Files algorithm\_params.py and helpers\_config.py tune parameters. RSRP thresholds. Cell filters. GDOP $\leq8.0$. Weights. Changes avoid core code edits. Path-loss works. Stage 2 formulas ($d = 10^y$) fit. Batch and hybrid modes run.

\subsubsection{pipeline phases:}

\begin{enumerate}
\item \textbf{Phase 1.1: Parse LTE logs}: G-NetTrack .txt to parsed\_*.csv. Add timestamps, GPS, RSRP, RSRQ, PCI, EARFCN.

\item \textbf{Phase 1.2: Tower coverage verification}: Match to towers.json/pci.json. Flag unmatched cells in reports.

\item \textbf{Phase 2.1: Signal extraction}: Clean RSRP/RSRQ to signaldata.csv, metadata.csv. Apply thresholds.

\item \textbf{Phase 2.2: Distance estimation}: Path-loss (Friis) or Stage 2 ($d = 10^y$). Context-specific.

\item \textbf{Phase 2.2.5: Distance ground-truth validation}: Haversine check. MAE, RMSE, CEP, R95, bias. Report output.

\item \textbf{Phase 2.3: RSRQ-Weighted bearings}: Bearings from signals and distances.

\item \textbf{Phase 3.1: Trilateration preparation}: 4--8 cells. GDOP $\leq$8.0. RSRP thresholds.

\item \textbf{Phase 3.2: Weighted Least-Squares solving}: Minimize $\sum w_i (d_{\text{measured}} - d_{\text{model}})^2$. $w_i = 1/\sigma_i^2$.

\item \textbf{Phase 3.3: Position ground-truth validation}: GPS match. MAE, RMSE, CEP, R95, bias, GDOP. Output accuracysummary.txt.
\end{enumerate}

\section{Evaluation metrics}

Metrics quantify GNSS and LTE positioning accuracy, distinguishing systematic bias from random errors per Chapter 2 taxonomy.

\subsection{GNSS (Stage 1)}
Haversine distances measure errors from GPS fixes to EXIF-ground-truth waypoints.
- \textbf{CEP}: 50th percentile radius containing half of errors.
- \textbf{R95}: 95th percentile radius for extremes.
- \textbf{RMSE}: Root mean square error against Garmin watch references.
Phone-watch agreement flags co-located positions within 10~m.

\subsection{LTE distance (Stage 2 \& Phase 2.2/2.2.5)}
Regression models report errors in log-space predictions ($y = 10\log_{10}(d_{\text{true}}$)).
- \textbf{MAE}: Mean absolute error.
- \textbf{RMSE}: Root mean square error.
- \textbf{MAPE}: Mean absolute percentage error.
Phase 2.2.5 adds bias detection for overfitting (e.g., systematic 90~m underestimation).

\subsection{LTE position (Phase 3.3)}
Trilateration outputs validate against GPS ground truth.
- \textbf{CEP}/\textbf{R95}/\textbf{RMSE}: As above, for 2D positions.
- \textbf{Bias}: Mean signed error, highlighting directional offsets.
- \textbf{GDOP}: Geometric Dilution of Precision distributions (target $\leq$8.0).

\section{Quality assurance}

Automated controls ensure pipeline reliability across GNSS and LTE stages, with logging for reproducibility in default/town/city/village contexts.

- \textbf{Clock synchronization}: Correct offsets $\leq$3600~s (1 hour, DST-tolerant) via optimal windows (0.5--8.0~s).
- \textbf{Data cleaning}: Flag GPS gaps ($\geq$12~s unstable, $\geq$60~s major, 30~s proximity); tower coverage verification with discrepancy reports.
- \textbf{Overfitting prevention}: LOFO cross-validation in Stage 2 rotates across files.
- \textbf{Geometry filtering}: Trilateration rejects GDOP $>$8.0, limits 4--8 cells.
- \textbf{Distance validation}: Phase 2.2.5 benchmarks MAE/RMSE vs. Haversine truths.
- \textbf{Dependency logging}: Phase-by-phase outputs (e.g., parsed\_*.csv, distanceestimates.csv) enable modular debugging.
- \textbf{Config Reproducibility}: Centralized params.py/algorithm\_params.py for environment-specific tuning.
